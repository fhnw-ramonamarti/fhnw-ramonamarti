\chapter{Einleitung}


\section{Problemstellung}

Im Web existieren diverse Möglichkeiten zur Erstellung eines Auswahl-Input mit vorgegebenen Werten.
Die HTML-Elemente bieten über die verschiedenen Browser keine konsistente Darstellung und Interaktion an.
Dazu sind sie nicht schön anzusehen und nur geringfügig anpassbar. 
Die Werte sind begrenzt auf Text und Unicode-Symbole.
Die Komponente ist nicht effizient bedienbar - besonders bei einer grossen Menge an Werten.

Als Alternative zu den Basis-Elementen existieren unzählige Bibliotheken, welche solche Eingabemöglichkeiten unterstützen.
Diese besitzen externe Abhängigkeiten, welche das eigene Projekt unnötig aufblasen.
Zudem benötigen viele dieser Libraries eine längere Einarbeitungszeit, um die Funktionalitäten verstehen und anwenden zu können.

Das Vorgängerprojekt deckt die Probleme ab, ist aber nur auf die Auswahl eines Landes zugeschnitten.
Eine Anwendung dieser Komponente mit anderen Inhalten kann zu unerwünschtem Verhalten führen.
Die oben genannten Probleme dienen als Basis für das folgende Kapitel.


\section{Ziel}

Die neue Komponenten zielt ab, die Auswahl eines Wertes von einer begrenzten, vorgegebenen Menge zu ermöglichen.
Dazu wird die vorangegangene Länderauswahl generalisiert und ausgebaut.
Die Eingabe soll weiterhin effizient bleiben und sich konsistent über alle Browser zeigen.
Dabei liegt der Fokus auf Browser von Desktop-Computern bzw. Laptops.
Als Werte sind nebst Texten auch Bilder wünschenswert.
Die Qualität ist auf dem Kolibri-Standard zu halten und durch Tests zu beweisen.
Der Nachweis der Anwendbarkeit wird über Usertests mit Programmierern als auch Anwendern gezeigt.
Automatisierte Komponententests stellen die Korrektheit der Implementation sicher.
Bei der Implementation wird darauf geachtet, dass die Design Patterns des Kolibri ihre Anwendung finden.
Dabei soll das Design und die Interaktion mit dem Toolkit synchronisiert sein.
Hierbei ist zu beachten, die Ziele nicht ausserhalb des Projekts zu definieren.


\section{Out of Scope}

Dieses Projekt bezieht sich auf die Entwicklung einer Auswahlkomponente für Nutzer ohne Seheinschränkungen.
Daher spielt die Accessibility nur eine begrenzte Rolle.
Screen-Reader müssen nicht beachtet werden, da dies zu umfangreich für diese Arbeit ist.
Die effiziente Anzeige von übergrossen Datenmengen mit mehr als 10'000 Werten ist nicht verlangt.
Die Eingabe eines eigenen Wertes in die neue Auswahlkomponente ist ebenfalls ausserhalb der Anforderungen.
Für die generalisierte Komponente reicht es, wenn die Auswahl eines einzelnen Wertes möglich ist.
Die Auswahl mehrerer Werte im selben Eingabe-Element ist nicht gefordert.
Die Komponente ist nicht speziell auf Mobile-Geräte auszurichten. 
Eine Undo- als auch eine Redo-Funktion der ausgewählten Werte ist ausserhalb des geforderten Rahmens. 

Ein Bestandteil dieser Arbeit ist das Erweitern des \texttt{SimpleInput}s um ein Select und eine Datalist.
Um den Fokus des Projekts auf der generalisierten Auswahlkomponente zu halten, sind keine Änderungen ausserhalb der oben genannten Ziele gefordert.
Anpassungen der Kern-Codebasis gehören nicht in den Rahmen dieses Projekts.
Durch die Eingrenzung der Anforderungen wird sichergestellt, dass die Ressourcen auf die Ziele fokussiert bleiben.


\section{Leitfaden}

Der Bericht umfasst die folgenden Kapiel: Hintergrund, existierende, neue Auswahlkomponenten und Schlusswort. 

{\color{red} \textbf{LEITFADEN KOMMT NOCH}}

% Das Konzeptionelle findet sich in den Kapiteln 2 bis 3.2 wieder. 
% Der Abschnitt 3.3 prüft das erarbeitete Konzept. 
% Dieses findet in nachfolgendem Unterkapitel 3.4 seine Umsetzung. 
% Das Resultat des Designs als auch der Implementation ist unter der Nummer 4 Länder-Auswahlkomponente beschrieben. 
% Kapitel 5 bewertet das Produkt und die erhaltenen Erkenntnisse.
