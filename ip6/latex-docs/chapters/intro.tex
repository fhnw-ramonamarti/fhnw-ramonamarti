\chapter{Einleitung}


\section{Problemstellung}

Im Web exestieren diverse Möglichkeiten zur Erstellung eines Auswahl-Input mit vorgegebenen Werten.
Die HTML-Elemente bieten über die verschiedenen Browser keine konsistente Darstellung und Interaktion an.
Dazu sind sie nicht schön anzusehen und nur sehr gering stylebar. 
Die Werte sind begrenzt auf Text und Unicode-Symbole.
Die Komponente ist nicht effizient bedienbar - besonders bei einer grossen Menge an Werten.

Als Alternative zu den Basis-Elementen existieren unzählige Bibliotheken, welche solche Eingabemöglichkeiten unterstützen.
Diese besitzen externe Abhänigkeiten, welche das eigene Projekt unnötig aufblasen.
Zudem benötigen viele dieser Libraries eine längere Einarbeitungszeit, um die Funktionalitäten verstehen und anwenden zu können.

Das Vorgängerprojekt deckt die Probleme ab, ist aber nur auf die Auswahl eines Landes zugeschnitten.
Eine Anwendung dieser Komponente mit anderen Inhalten kann zu unerwünschten Verhalten führen.
Die oben genannten Probleme dienen als Basis für das folgende Kapitel.


\section{Ziel}

Die neue Komponenten zielt ab, die Auswahl eines Werten von einer begrenzten, vorgegeben Menge zu ermöglichen.
Dazu wird die voran gegangene Länderauswahl generalisiert und ausgebaut.
Die Eingabe soll weiterhin effizient bleiben und sich konsistent über alle Browser zeigen.
Als Werte sind nebst Texten auch Bilder wünschenswert.
Die Qualität ist auf dem Kolibri-Standard zu halten und durch Tests zu beweisen.
Der Nachweis der Anwendbarkeit wird über Usertests mit Programmierern als auch Anwendern gezeigt.
Automatisierte Komponententests stellen die Korrektheit der Implementation sicher.
Bei der Implementation wird darauf geachtet, dass die Design Patterns des Kolibri ihre Anwendung finden.
Dabei soll das Design und die Interaktion mit dem Toolkit synchronisiert sein.
Hierbei ist zu beachten, die Ziele nicht ausserhalb des Projekts zu definieren.


\section{Out of Scope}

Dieses Projekt bezieht sich auf die Entwicklung einer Auswahlkomponente für Nutzer ohne Seheinschränkungen.
Daher spielt die Accessability nur eine begrenzte Rolle.
Screen-Reader müssen nicht beachtet werden, da dies zu umfangreich für diese Arbeit ist.
Die Eingabe eines eigenen Wertes in die neue Auswahlkomponente ist ebenfalls ausserhalb der Anforderungen.
Für die generalisierte Komponente reicht es, wenn die Auswahl eines einzelnen Wertes möchlich ist.
Die Auswahl mehrer Werte im selben Eingabe-Elemente ist nicht gefordert.
Weiter ist eine Undo- als auch ein Redo-Funktion der ausgewählten Werte ausserhalb des geforderten Rahmens. 

Um den Fokus des Projekts auf der generalisierten Auswahlkomponente zu halten, sind keine Änderungen ausserhalb des Rahmens gefordert.
Anpassungen der Kern-Codebasis gehören nicht in den Rahmen dieses Projekts.
Das Erweitern des SimpleInput um die Möglichkeiten eines Auswahl-Inputs befindet innerhalb der Arbeit.
Durch die Eingrenzung der Anforderungen wird sichergestellt, dass die Ressourcen auf die Ziele aufgeteilt werden können.


\section{Leitfaden}

Der Bericht teilt sich in die Teile Grundlagen, existierende und neue Komponente. 


% Das Konzeptionelle findet sich in den Kapiteln 2 bis 3.2 wieder. 
% Der Abschnitt 3.3 prüft das erarbeitete Konzept. 
% Dieses findet in nachfolgendem Unterkapitel 3.4 seine Umsetzung. 
% Das Resultat des Designs als auch der Implementation ist unter der Nummer 4 Länder-Auswahlkomponente beschrieben. 
% Kapitel 5 bewertet das Produkt und die erhaltenen Erkenntnisse.
