\chapter{Einleitung}
\label{chap:intro}


\section{Problemstellung}
\label{sec:problem}

Im Web existieren diverse Möglichkeiten zur Erstellung eines Auswahl-Input mit vorgegebenen Werten.
Die HTML-Elemente bieten über die verschiedenen Browser keine konsistente Darstellung und Interaktion an.
Dazu sind sie nicht schön anzusehen und nur geringfügig anpassbar. 
Die Werte sind begrenzt auf Text und Unicode-Symbole.
Die Komponente ist nicht effizient bedienbar - besonders bei einer grossen Menge an Werten.

Als Alternative zu den Basis-Elementen existieren unzählige Bibliotheken, welche solche Eingabemöglichkeiten unterstützen.
Diese besitzen externe Abhängigkeiten, welche das eigene Projekt unnötig aufblasen.
Zudem benötigen viele dieser Libraries eine längere Einarbeitungszeit, um die Funktionalitäten verstehen und anwenden zu können.

Das Vorgängerprojekt deckt die Probleme ab, ist aber nur auf die Auswahl eines Landes zugeschnitten.
Eine Anwendung dieser Komponente mit anderen Inhalten kann zu unerwünschtem Verhalten führen.
Die oben genannten Probleme dienen als Basis für das folgende Kapitel.


\section{Ziel}
\label{sec:goal}

Die neue Komponenten zielt ab, die Auswahl eines Wertes von einer begrenzten, vorgegebenen Menge zu ermöglichen.
Dazu wird die vorangegangene Länderauswahl generalisiert und ausgebaut.
Die Eingabe soll weiterhin effizient bleiben und sich konsistent über alle Browser zeigen.
Dabei liegt der Fokus auf Browser von Desktop-Computern bzw. Laptops.
Als Werte sind nebst Texten auch Bilder wünschenswert.
Die Qualität ist auf dem Kolibri-Standard zu halten und durch Tests zu beweisen.
Der Nachweis der Anwendbarkeit wird über Usertests mit Programmierern als auch Anwendern gezeigt.
Automatisierte Komponententests stellen die Korrektheit der Implementation sicher.
Bei der Implementation wird darauf geachtet, dass die Design Patterns des Kolibri ihre Anwendung finden.
Dabei soll das Design und die Interaktion mit dem Toolkit synchronisiert sein.
Hierbei ist zu beachten, die Ziele nicht ausserhalb des Projekts zu definieren.


\section{Out of Scope}
\label{sec:outOfScope}

Dieses Projekt bezieht sich auf die Entwicklung einer Auswahlkomponente für Nutzer ohne Seheinschränkungen.
Daher spielt die Accessibility nur eine begrenzte Rolle.
Screen-Reader müssen nicht beachtet werden, da dies zu umfangreich für diese Arbeit ist.
Die effiziente Anzeige von übergrossen Datenmengen mit mehr als 10'000 Werten ist nicht verlangt.
Die Eingabe eines eigenen Wertes in die neue Auswahlkomponente ist ebenfalls ausserhalb der Anforderungen.
Für die generalisierte Komponente reicht es, wenn die Auswahl eines einzelnen Wertes möglich ist.
Die Auswahl mehrerer Werte im selben Eingabe-Element ist nicht gefordert.
Die Komponente ist nicht speziell auf Mobile-Geräte auszurichten. 
Eine Undo- als auch eine Redo-Funktion der ausgewählten Werte ist ausserhalb des geforderten Rahmens. 

Ein Bestandteil dieser Arbeit ist das Erweitern des \texttt{SimpleInput}s um ein Select und eine Datalist.
Um den Fokus des Projekts auf der generalisierten Auswahlkomponente zu halten, sind keine Änderungen ausserhalb der oben genannten Ziele gefordert.
Anpassungen der Kern-Codebasis gehören nicht in den Rahmen dieses Projekts.
Durch die Eingrenzung der Anforderungen wird sichergestellt, dass die Ressourcen auf die Ziele fokussiert bleiben.


\section{Leitfaden}
\label{sec:tocTexted}

Dieser Bericht gliedert sich in die Teile \textbf{\nameref{chap:background}}, \textbf{\nameref{chap:existingComponents}} und \textbf{\nameref{chap:newComponent}}.
Jedes Kapitel baut auf dem vorherigen auf und führt den Leser Schritt für Schritt durch die Entwicklung und Optimierung der neuen Auswahlkomponente.

Im Kapitel \textbf{\nameref{chap:background}} ist die \textbf{\nameref{sec:basics}} des Kolibri-Toolkits und des Projekts erläutert.
Es folgt eine Erklärung der verschiedenen \textbf{\nameref{sec:states}}, die eine Auswahlkomponente haben kann, und eine Beschreibung der populärsten Browser und deren Rendering-Engines.
Ein detaillierter Ablauf des Parsens und Renderns einer HTML-Seite sowie die wichtigsten Browser-Implementationen sind ebenfalls auf dem Plan.

Das Kapitel \textbf{\nameref{chap:existingComponents}} vergleicht die HTML \texttt{datalist} und \texttt{select} Komponenten, beschreibt die Nutzung und Unterschiede der verschiedenen Tags, 
und hebt Browser-Inkonsistenzen hervor, die zu verschiedenen UI-Erfahrungen führen können.

Die \textbf{\nameref{chap:newComponent}} wird im Detail beschrieben.
Der \textbf{\nameref{sec:design}}-Ansatz basiert auf dem Kolibri-Designsystem, unterstützt durch Figma-Prototypen zur Visualisierung und zum Testen.
Das \textbf{\nameref{sec:interactionDesign}} optimiert die Benutzerführung für Maus- und Tastaturbenutzer.
\textbf{\nameref{sec:userFeedbackPtototyping}} in einem iterativen Prozess tragen kontinuierlich zur Verbesserung bei.

Im Abschnitt \textbf{\nameref{sec:interaction}} sind Regeln für die Maus- und Tastaturinteraktion festgelegt.
Die Einhaltung diverser \textbf{\nameref{sec:principleRules}} sorgt für stabilen und verständlichen Code.
Der Einsatz von \textbf{\nameref{sec:patterns}} wie Null-Object, Projector und Decorator strukturiert die Implementation und erhöht die Wartbarkeit des Codes.

Das Kapitel \textbf{\nameref{sec:performance}} beschreibt Optimierungen zur Verbesserung der Ladezeit und Performance der neuen Auswahlkomponente.
Das \textbf{\nameref{sec:testing}}-Kapitel dokumentiert die Durchführung automatisierter Tests sowie die Auswertung von Usability-Tests - aus Sicht von Programmierern als auch Endnutzern.
Schliesslich bietet die \textbf{\nameref{sec:discussion}} der Ergebnisse einen Überblick über die Bedeutung der Erkenntnisse für die Entwicklung 
und das \textbf{\nameref{sec:summeryNew}} fasst die wichtigsten Erkenntnisse zusammen und gibt einen Ausblick auf zukünftige Arbeiten.
