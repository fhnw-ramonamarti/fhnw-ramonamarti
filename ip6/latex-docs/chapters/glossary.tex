\chapter{Glossar}
\label{chap:glossary}

\newcommand{\glossarywithTitle}{0.22\textwidth}
\newcommand{\glossarywith}{0.76\textwidth}

\vspace*{-1cm}
\begin{table}[!ht]
    \centering
    \rowcolors{2}{white}{gray!20}
    \footnotesize
    \begin{adjustbox}{max width=\textwidth}
        \begin{threeparttable}
            \begin{tabular}{ p{\glossarywithTitle} | p{\glossarywith} }
                \bf{Begriff} & \bf{Beschreibung} \\
                \hline \hline
                \bf{Aktuelle Browser} & \tbbr{
                    Edge: Version 127 (Windows) \\
                    Chrome: Version 127 (Windows / Mac) \\
                    Firefox: Version 128 (Windows / Mac) \\
                    Safari: Version 17.5 (Mac)
                } \\
                \hline
                \bf{ARIA-Rolle} & Accessible Rich Internet Applications Role.
                    Sie definiert die Bezeichnung oder Funktion eines HTML-Elements.
                    Sie dient als Markierung für Struktur und erweitert die Bedienbarkeit für u.a. Screenreader. \\
                \hline
                \bf{Ausgrauen} & Das Element erscheint in Grautönen mit eher schwachem Kontrast. \\
                \hline
                \bf{Client-Side} & Direkt im Browser. 
                    Aktionen sehen den Server nie und bleiben im Browser. 
                    Der Code ist direkt auf der HTML-Seite eingebunden. 
                    In Zusammenhang mit Formularen erfährt der Server nichts von der Client-Side Validierung. \\
                \hline
                \raggedright \bf{Do-Nothing-Implementation} & 
                    Implementation, welche als Stellvertreter z.B. beim Null-Object Pattern dient. 
                    Eine Funktion mit einer solchen Implementation führt nichts aus und hat keine Seiteneffekte. \\
                \hline
                \bf{Endanwender} & Ein Nutzer, der die neue Komponente ausfüllt. \\
                \hline
                \bf{Hovern} & Mit der Maus über ein Element fahren. \\
                \hline
                \bf{Immutable} & Unveränderbar. Objekte sind nicht überschreibbar. \\
                \hline
                \bf{Kategorie} & Gruppierung der Optionen. 
                    Das Selektieren einer Kategorie in einer \codestyle{Select\-Component} reduziert die Anzahl der Optionen. 
                    Die Komponente zeigt nur noch Optionen an, welche der Kategorie / Gruppe zugewiesen sind. \\
                \hline
                \bf{KISS} & Keep it Simple and Stupid. 
                    Ein Prinzip der Programmierung, in welcher alles möglich einfach zu halten ist. 
                    Einfache und kurze Codestücke minimieren das Risiko von Fehlern und Unverständlichkeit. \\
                \hline
                \bf{Multiselect} & 
                    Auswahlkomponente (HTML \codestyle{select} Element), bei welchem mehrere Werte selektierbar sind. 
                    Bei der Selektion einer zweiten Option hebt sich die vorherige Auswahl nicht auf. 
                    Auf Desktop-Browsern muss jedoch die passende Taste bei der Auswahl gedrückt sein, 
                        damit sich die Selektion erweitert. \\
                \hline
                \bf{Regex} & Regular Expression. 
                    Beschreibt eine Zeichenkette. 
                    Syntaktische Regeln helfen bei der Verarbeitung von Texten. \\
                \hline
                \raggedright \bf{Separation of Concern} & 
                    Ein Prinzip, welches in der Programmierung beim Aufteilen eines Programmes zur Anwendung kommt. 
                    Jede Komponente besitzt nur eine Aufgabe. 
                    Dadurch entsteht ein modularer Aufbau. \\
                \hline
                \bf{Single-Select} & 
                    Auswahlkomponente (HTML \codestyle{select} Element), bei welchem nur ein Wert selektierbar ist. 
                    Bei der Selektion einer zweiten Option hebt sich die vorherige Auswahl auf. \\
                \hline
                \bf{Spiegelstrich} & 
                    Ein Strich, welcher sich auf der linken Seite eines Elements bzw. einer Option befindet. 
                    Der Strich ist vertikal und dient zum Hervorheben des Elements. \\
                \hline
                \bf{Triggern} & Auslösen eines Events. \\
                \hline
                \bf{UI/UX-Design} & Design der Benutzeroberfläche und der Benutzererfahrung. \\
                \hline
                \raggedright \bf{Vorarbeit/ vorangegangenes Projekt} & 
                    Semesterarbeit bevor die Bachelor-Arbeit beginnt. 
                    In dieser Arbeit finden Vorbereitungen auf die kommende Bachelor-Arbeit statt. \\
                \hline
                \bf{Workaround} & 
                    Alternativer Code, wenn der Ursprüngliche nicht funktioniert. 
                    Eine Möglichkeit ein Problem zu umgehen. \\
                \hline
                \hline
                \raggedright \bf{* Tabellen-Fussnote} & Bemerkung gilt für die ganze Tabelle. \\
                \hline
            \end{tabular}
        \end{threeparttable}
    \end{adjustbox}
\end{table}
