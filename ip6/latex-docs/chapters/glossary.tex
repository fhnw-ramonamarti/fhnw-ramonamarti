\chapter{Glossar}
\label{chap:glossary}

\newcommand{\glossarywithTitle}{0.32\textwidth}
\newcommand{\glossarywith}{0.66\textwidth}
\begin{table}[!ht]
    \smallskip 
    \centering
    \rowcolors{2}{white}{gray!20}
    \footnotesize
    \begin{adjustbox}{max width=\textwidth}
        \begin{threeparttable}
            \begin{tabular}{ p{\glossarywithTitle} | p{\glossarywith} }
                \bf{Begriff} & \bf{Beschreibung} \\
                \hline \hline
                \bf{Client-Side} & Direkt im Browser. 
                    Aktionen sehen den Server nie und bleiben im Browser.
                    Der Code ist direkt auf der HTML-Seite eingebunden.
                    In Zusammenhang mit Formularen erfährt der Server nichts von der Client-Side Validierung. \\
                \hline
                \bf{Do-Nothing-Implementation} & 
                    Implementation, welche als Stellvertreter z.B. beim Null-Object Pattern dient. 
                    Eine Funktion mit einer solchen Implementation führt nichts aus und hat keine Seiteneffekte. \\
                \hline
                \bf{Hovern} & Mit der Maus über ein Element fahren. \\
                \hline
                \bf{Immutable} & Unveränderbar. Objekte sind nicht überschreibbar. \\
                \hline
                \bf{Kategorie} & Gruppierung der Optionen. 
                    Das Selektieren einer Kategorie in einer \codestyle{SelectComponent} reduziert die Anzahl der Optionen.
                    Die Komponente zeigt nur noch Optionen an, welche der Kategorie / Gruppe zugewiesen sind. \\
                \hline
                \bf{Multiselect} & 
                    Auswahlkomponente (HTML \codestyle{select} Element), bei welchem mehrere Werte selektierbar sind. 
                    Bei der Selektion einer zweiten Option hebt sich die vorherige Auswahl nicht auf.
                    Auf Desktop-Browsern muss jedoch die passende Taste bei der Auswahl gedrückt sein, 
                        damit sich die Selektion erweitert. \\
                \hline
                \bf{Regex} & Regular Expression. 
                    Beschreibt eine Zeichenkette.
                    Syntaktische Regeln helfen bei der Verarbeitung von Texten. \\
                \hline
                \bf{Separation of Concern} & 
                    Ein Prinzip, welches in der Programmierung beim Aufteilen eines Programmes zur Anwendung kommt.
                    Jede Komponente besitzt nur eine Aufgabe. 
                    Dadurch entsteht ein modularer Aufbau. \\
                \hline
                \bf{Single-Select} & 
                    Auswahlkomponente (HTML \codestyle{select} Element), bei welchem nur ein Wert selektierbar ist. 
                    Bei der Selektion einer zweiten Option hebt sich die vorherige Auswahl auf. \\
                \hline
                \bf{Spiegelstrich} & 
                    Ein Strich, welcher sich auf der linken Seite eines Elements bzw. einer Option befindet.
                    Der Strich ist vertikal und dient zum Hervorheben des Elements. \\
                \hline
                \bf{Triggern} & Auslösen eines Events. \\
                \hline
                \bf{UI/UX-Design} & Design der Benutzeroberfläche und der Benutzererfahrung. \\
                \hline
            \end{tabular}
        \end{threeparttable}
    \end{adjustbox}
\end{table}
