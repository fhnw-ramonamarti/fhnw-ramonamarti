\chapter{Neue Auswahlkomponenten}

\section{Design}

\subsection{Mögliche Designoptionen eines Elements}

% border, background-color, font-color, underline, italic, font-weight, line left side
% pro cons pro möglichkeit mit passenden farboptionen
% kombination 2er status eines elements


\section{Interaktionen}

Für die Bedienung der Komponente gilt es Regeln festzulegen, damit ein gemeinsames Verständis entsteht.
Wie in den Grundlagen bereits beschrieben kann sich ein Wert aus der Komponentenliste verschiedene in Zuständen befinden.
In diesem Absatz spielen Selektion, Highlight und Cursor Position eine Rolle.
Zur Auffrischung: 

\begin{itemize}
    \item \textbf{Selektion}: Ausgewählter Wert der Spalte
    \item \textbf{Highlight}: Element unterhalb des Maus Zeiger
    \item \textbf{Cursor Position}: Element-Position für die Tastatur
\end{itemize}

\noindent
Bei der Festlegung der Maus-Interaktion fiel die Entschiedung auf folgendes:

\begin{itemize}
    \item \textbf{mouseover}: visuelles Highlighting des Elements ohne Selektionsänderung
    \item \textbf{click}: wie mit der Tastatur
\end{itemize}

\noindent
Hingegen die Tastatur-Steuerung mit den Pfeiltasten hält sich an diese Bedienungen:

\begin{itemize}
    \item direkte Selektionsänderung ohne weiter Bestätigung
    \item Änderung der Cursor Position
\end{itemize}

\noindent
Anhand dieser Regeln enstanden folgendene Aktionen als Basis für den ersten Projektor der neuen Komponente. 


\clearpage
\import{../tables}{d.newComponent.tex}

Das Undo und das Redo auf der Komponente erhält im ersten Projektor keine speziell Definition.
% Leerschlag & Buchstaben
Die Interaktionen können in weiteren Pojektoren angepasst bzw. geändert werden.



\section{Umsetzung}

% ...


\subsection{Prinzipien}

% immutable ist besser, clean coding rules, kiss (simple stupid) vorgehen, inkrementelle entlickung, separation of concern
% eintscheidung treffen, bewusst weglassen, stand der technik


\subsection{Patterns}

% decorator pattern, projector pattern, null object pattern


\subsection{Popover}

% popover api 2024
% dialog popup

