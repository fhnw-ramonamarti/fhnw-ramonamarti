\chapter{Existierende Auswahlkomponenten}

\import{../tables}{c.choice.general.tex}


\section{HTML Datalist vs. Select}

\subsection{Datalist}


\subsection{Select}




\section{Browser-Inkonsistenzen}
Die Komponenten Select \& Datalist können sich im UI und der Bedingung leicht varieren,
besonders hinsichtlich der visuellen Darstellung und des Verhaltens von Navigationspfeilen oder anderen Steuerelementen.

\clearpage
\subsection{Edge Browser}
\import{../tables}{c.edge.tex}
Auf Windows wehält sich Edge sehr ähnlich wie Chrome, da die Codebasis beider Browser Chromium ist.
Im Design gibt es jedoch Unterschiede bei der Darstellung von den Listen oder Navigationspfeilen. 

\clearpage
\subsection{Chrome Browser}
\import{../tables}{c.chrome.tex}

Auf dem Mac verhält sich Chrome ähnlich wie auf Windows, jedoch können sich Designaspekte unterscheiden. 
Zum Beispiel könnte der Navigationspfeil in einem anderen Stil dargestellt werden oder 
die Animationen beim Öffnen von Dropdowns könnten glatter sein.

\clearpage
\subsection{Firefox Browser}
\import{../tables}{c.firefox.tex}

Firefox zeigt auf dem Mac konsistentes Verhalten wie auf Windows, jedoch mit typischen MacOS Designanpassungen. 
Elemente wie Buttons und Listen könnten abgerundetere Ecken haben und 
die Interaktions-Feedbacks können sich leicht von der Windows-Version unterscheiden.

\clearpage
\subsection{Safari Browser}
\import{../tables}{c.safari.tex}

Ähnlich wie in Chrome und Firefox auf Windows, aber Designelemente wie Schatten oder Ränder können subtiler sein. 
Safari tendiert dazu, ein minimalistischeres Design zu verwenden, was sich in der Darstellung von Dropdowns und Comboboxen widerspiegelt.


\clearpage
\subsection{Fazit}

Generell ändern sich die Verhaltensweisen der Standardkomponenten in verschiedenen Browsern in Bezug auf Design und Interaktionsmechanismen. 
Hierbei spielen das Betriebssystem-spezifische Styling und die individuellen Implementierungen der Browser eine Rolle. 
Die Grundfunktionen zeigen sich ähnlich, wobei etwaige Unterschiede oft in der visuellen Darstellung und in Details der Interaktion 
(z.B. beim Öffnen der Dropdowns oder der Navigation innerhalb der Listen) liegen. 
Wichtig ist somit die Konsistenzprüfung, um sicherzustellen, dass Webanwendungen über verschiedene Plattformen
und Browser hinweg einheitlich funktionieren und eine optimale Benutzererfahrung bieten. 
Das Verhalten kann je nach Browserversion und Betriebssystem variieren.

\section{Anwendungsfälle}
\begin{itemize}
    \item Länderauswahl (nach Kontinenten filtern)
    \item Jahrgang auswählen (nach Dekade filtern)
\end{itemize}
