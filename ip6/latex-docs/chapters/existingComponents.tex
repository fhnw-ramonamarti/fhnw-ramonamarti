\chapter{Existierende Auswahlkomponenten}

Dieses Kapitel zeigt einen Überblick als auch Vergleiche der bestehenden Optionen eine Auswahl zu treffen. 
Es werden die generellen Möglichkeiten einander gegenübergestellt, sowie die bisher bekanntesten Komponenten $select$ und $datalist$ genauer inspiziert.
Darin wird auf die Interaktionen und die Darstellung bezug genommen. 

Zu den bereits existierenden Komponenten zum Start dieser Arbeit zählen Buttons bzw. Links, $select$-, $datalist$-Element und das Country Select.
Unter dem Country Select ist das Resultat des Projekt 5 zu verstehen.
In der folgende Tabelle \ref{table:generalComparing} zeigt einen Vergleich der genannten Möglichkeiten.

\import{../tables}{c.choice.general.tex}

\section{HTML Datalist vs. Select}

Die folgenden Unterkapitel zeigen die Möglichkeiten auf, welche die HTML-Elemente $input$, $option$, $datalist$ und $select$ bieten.
Zudem zeigen tabellarische Gegenüberstellungen die Unterschiede und Inkonsistenzen dieser in verschiedenen Browsern und Betriebssystemen auf.
Hierbei liegt der Fokus mehr auf der Interaktion mit den Komponenten als auf der Darstellung.


\subsection{Option}

\begin{lstlisting}[language = html, caption = Code: Option Example, label = code:OptionExample]
    <option value="ValueToSend">LabelToShow</option>
    <option value="DefaultValue" selected>DefaultLabel</option>
    <option value="Value" disabled>DisabledLabel</option>
\end{lstlisting}

Das $option$-Tag wird bei beiden Auswahlkomponenten - $datalist$ und $select$ - verwendet. 
Die einzelnen Optionen wie in Code \ref{code:OptionExample} Zeile 1 besitzen ein $value$-Attribut und einen Inhalt - das Label.
Der Wert kann als Inhalt verwendet werden, wenn anstatt das $value$- das $label$-Attribut genutzt wird.
Das $label$ wird jedoch von Firefox nicht unterstützt.
Werden $value$ und $label$ weggelassen, erhalten beide den Wert innerhalb des Tags.
Das HTML-Element kann das Attribut $disabled$ oder $selected$ erhalten.
Eine $selected$ Option wie in Code \ref{code:OptionExample} Zeile 2 wird initial ausgefüllt und funktioniert nur im Zusammenhang mit dem $select$-Container. 
Ist ein Eintrag $disabled$ wie in Code \ref{code:OptionExample} Zeile 3 kann dieser nicht ausgewählt werden und unterscheidet sich im UI, wenn dieser überhaupt sichtbar ist.

Das Designen von $option$s beschränkt sich auf die Text- / Hintergrundfarbe, welche jedoch nur in Firefox funktionieren. 
Die anderen Browser lassen kein Styling der Auswahl-Elemente zu bzw. zeigen diese nicht an.
Es ist nur ein scalarer Text als Inhalt erlaubt. 
Das bedeutet, es können keine weiteren Tags verschachtelt werden.
Als Eltern-Elemente sind $select$, $optgroup$ und $datalist$ erlaubt.
Alle gängigen Browser unterstützen dieses Tag.



\subsection{Select}

\begin{lstlisting}[language = html, caption = Code: Select Example, label = code:SelectExample]
    <select name="animal">
        <option value="" disabled>Animal</option>
        <option value="dog">Dog</option>
        <option value="cat">Cat</option>
        <option value="mouse">Mouse</option>
    </select>
\end{lstlisting}

Das $select$-Tag benötigt kein weiteres Element ausser die $option$s – ersichtlich im Code \ref{code:SelectExample}, um den Wert zu speichern.
Die typischen Attrribute für Eingaben-Felder wie $disabled$, $form$, $name$ und $required$ sind auch bei einem $select$ verwendbar.
Durch das $disabled$ zeigt sich das Element in einer ausgegrauten Erscheinung und bietet dem Nutzer keine Interaktionsmöglichkeiten mehr.
Dieses Attribut kann vom Container - z.B. $fieldset$ - geerbt werden.
Das $form$ definiert das dazugehörige Formular und $name$ den Key für das Key-Value Paar bzw. Name des Feldes im Formular. 
Mit dem $required$ wird eine Eingabe erzwungen, um das senden des Formulars freizuschalten.
Für eine bessere Accessability sollte eine $id$ vergeben werden, damit ein Label dem $select$ zugewiesen werden kann.
Standardmässig ist für dieses Feld nur eine einzelne Auswahl möglich.
Durch die Ergänzung des Attribut $multible$ ist es möglich mehrere Werte zu markieren.
Bei einem Multiselect können mehrere Listenelemente mit vorausgewählt werden.
Wenn jedoch bei einem einfachen $select$ mehrere Optionen das $selected$-Attribut enthalten, wird das letzte als Default eingesetzt und die anderen ignoriert.
Das $autocomplete$ funktioniert wie bei anderen Inputs, indem Hinweise für das User Agent Feature auftauchen.
Um nach dem Laden der Seite direkt den Fokus auf das Feld zu setzen, kann das Attribut $autofocus$ ergäntzt werden.
Die Definition des $size$-Attributs steuert die Anzahl sichtbarer Elemente, wobei der Default für single $select$s bei 1 und mehrfache Auswahl bei 4 liegt.
Das $select$ erhält von allen allen Browsern Unterstützung mit Ausnahme des $size$-Attributs.
Die Grösse wird auf den meisten Mobile-Browsern nicht unterstützt.

Abgesehen von der Grösse durch die $size$ kann das Element kaum umgestaltet werden.
Es gibt jedoch die sehr aufendiwge Möglichkeit, den Inhalt zu klonen und durch Wrapper neu zu stylen.
In diesem Fall müssen aber die Logik und die Interaktionen für die Accessability neu implementiert werden.
Gewisse Stylings können getroffen werden, wobei aber nicht jeder Browser diese in der selben Weise übernehmen.
Durch die komplexe Strukutr des $selet$s ist eine eigene Darstellung schwierig zu kontrollieren.

Der Inhalt des Tags können $option$s oder $optgroup$s sein.
Mehr zur $optgroup$ kommt im nachfolgenden Unterkapitel.
In der ARIA-Rolle wird die Komponente als Combobox ohne $size$ und $multible$ oder Listbox gehandhabt.

Die mehrfache Auswahl rein per Tastatur bietet in der Interakion geringer Möglichkeiten als mit der Maus.
Mit gedrückter Cmd bzw. Ctrl (Mac bzw. Windows) Taste können mit der Maus weitere Elemente dazugewählt werden.
Sollen in einem Beriech alle Elemente zwischen dem ersten und dem letzten ebenfalls markiert werden, kann stattdessen Shift gedrückt gehalten werden.
Wenn beide Techniken zur Anwendung kommen, muss zuerst der Teil mit Shift gewählt werden, da sonst die vorher gehende Auswahl aufgehoben wird.
Um nur mit der Tastatur mehrere Werte zu markieren, muss zuerst zum ersten Element navigiert werden.
Mit Shift und den Pfeiltasten hoch und runter ist ein Bereich wählbar.
Firefox unterstützt noch die einzelne Mehrfachauswahl über die Tastatur, indem mit Ctrl der Focus hoch und runter bewegt wird und mit der Leertaste selektiert werden kann.
Auf dem Mac müssen die Tastenkombinationen jedoch zuerst in der Systemeinstellung ausgeschaltet oder umgestellt werden, da es Systembefehle sind.


\subsubsection{Optgroup}

\begin{lstlisting}[language = html, caption = Code: Optgroup Example, label = code:OptgroupExample]
    <select name="animal">
        <optgroup label="Big animal">
            <option value="dog">Dog</option>
            <option value="cat">Cat</option>
        </optgroup>
        <optgroup label="Small animal">
            <option value="mouse">Mouse</option>
            <option value="hamster">Hamster</option>
        </optgroup>
    </select>
\end{lstlisting}

Die $optgroup$ dient als Zusatz-Element um in $select$s Optionen zu Gruppieren und mit einem Zwischentitel zu versehen.
Der Titel ist nicht selektierbar und lässt sich durch das $label$-Attribut setzen. 
Um ein ganzer Block von Auswahlmöglichkeiten auszugrauen, bietet sich das $disabled$ an.
Die Optionen, welche in diesem Tag enthalten sind, erben das Attribut.
Als Inhalt dienen $option$-Elemente und selbst besitzt es als Eltern-Tag ein $select$.
Es besitzt die ARIA-Rolle einer Gruppe und wird von allen Browsern unterstützt.


\subsection{Datalist}

\begin{lstlisting}[language = html, caption = Code: Datalist Example, label = code:DatalistExample]
    <input type="text" name="animal" list="data" placeholder="Animal" />
    <datalist id="data">
        <option value="Dog"></option>
        <option value="Cat"></option>
        <option value="Mouse"></option>
    </datalist>
\end{lstlisting}

Die Datalist besteht aus zwei Teilen - einem Input-Feld und einem Daten-Container - wie in Code \ref{code:DatalistExample} ersichtlich. 
Im folgenden Unterkapitel \textbf{Input} wird detaillierter auf das Eingabe-Feld und dessen Typen eingegangen.
Die Datenliste besitzt keine spezeillen eigenen Attribute.
Von den globalen Attributen sollte jedoch zumindest eine $id$ vergeben werden.
Die Id dient zur Verknüpfung der Liste mit dem Input-Feld. 
Das Stylen der $datalist$ ist sehr begrenzt bzw. nicht möglich. 
Der Daten-Container reagiert nicht auf den Zoom des Browsers.
Gewisse Screenreader ingorieren die Vorschlagsliste und lesen diese somit nicht vor.
In der ARIA-Rolle wird das Element als Listbox interpretiert.

Die $option$s einer $datalist$ besitzen normalerweise nur ein $value$-Attribut und kein Label oder nur Inhalt.
Bei einer zusätzlichen Label-Definition kann das je nach Browser zu abweichenden Darstellungen kommen. 
Während Firefox nur das Label in der Liste anzeigt, werden in den anderen Browsern Label und Value gemeinsam visualisiert. 
Die Browser, welche Label und Value anzeigen, heben den Wert ein wenig hervor.
Die Darstellung kann zu Missverständissen führen, da bei der Auswahl nur der Value in das Eingabe-Feld eingefügt wird. 
Die $options$-Werte können sich dem Typ des Inputs anpassen. 
Generell unterstützen alle Browser die $datalist$, aber Firefox nur begrenzt.

(\cite{optionMdn}) Nicht alle Browser unterstützen die Liste für jeden Eingabe-Typ.
Der textuelle Typ funktioniert in allen Browsern und die Liste öffnet sich nach ein Klick bzw. Doppelklick auf das Feld.
Vordefinierte Datum- und Zeit-Typen funktionieren nur in den Chromium basierten Browsern. 
Die anderen Browser - Firefox und Safari - zeigen den normalen Eingabe-Container, als ob die Liste nicht verknüpft ist.
Wenn eine Liste mit einer Range verwendet wird, zeigen alle Browser die Optionen durch Markierungen an der jeweiligen Stelle auf dem Slider.
Die Kombination einer $datalist$ mit einer Farbpalette zeigt eine breite, aber unterschiedliche Unterstützung. 
Firefox auf OSX ist unter den gängigen Browser der einzige, welche die Liste nicht darstellt.
Zusammengefasst bieten Chromium-basierte Browser die beste Unterstützung für Listen mit den unterschliedlichten Typen.


\subsubsection{Input}

Dieser Abschnitt behandelt nur den Teil, welcher im Bezug auf die $datalist$ von Belang ist.
Das Input - in Code \ref{code:DatalistExample} auf Zeile 1 - wird durch das $list$-Attribut mit den Auswahloptionen verknüpft.
Bei der Verwendung zusammen mit der $datalist$ ist dieses Attribut pflicht.
Dadurch erscheinen die Auswahloptionen während der Bedienung des Input-Feldes. 

Die in diesem Kontext für alle Typen geltenden Attribute werden in diesem Paragraph genauer angeschaut.
Mit dem $autocomplete$ dient zur Anzeige der Hinweise für das Autofill Featuere der Browser.
Wie in den früheren Abschnitten bereits beschrieben kann durch das $disable$ die Interaktion ausgeschaltet werden und somit die Komponente deaktiviert werden.
Wenn das Eingabe-Feld ausserhalb eines Formulars platziert ist, kann durch das Attribut $form$ ein Verknüpfung zu einem auf der Seite existierenden Formular hergestett werden.
Der Name mit $name$ dient zur Identifizierung des definierten Wertes im $value$-Attribut in einem abgesendet Formular.
Der Typ des Eingabefeldes - angegeben durch das $type$ - definiert, welche UI-Erscheinung das Input im Browser erhält und welche Werte zulässig sind.
Als Standard-Typ ist $text$ definiert, wodurch dieses Attribut optional ist.
Mehr zu den für diese Arbeit wichtigen Typen ist im Unterkapitel \textbf{Input-Typen mit Datalist-Unterstützung} zu lesen.
Die $id$ als globale Eigenschaft dient bei den Eingabe-Felder zusätzlich noch zur Verkünpfung mit einem $label$-Element und somit einer bessern Accessability.

Weiter existieren noch die zwei Attribute $readonly$ und $required$, welche alle Typen ausser $range$ und $color$ definiert sind.
Durch die Ergänzung der ersten Eigenschaft ist der Wert nicht mehr änderbar, aber es bleibt im Verlauf der fokussierbaren Komponenten enthalten.
Um die Eingabe zu erzwingen, kann das $required$ angewendet werden.
Die Eigenschaften $min$, $max$ und $step$ helfen numerischen Typen\footnotemark zu konfigurieren.
\footnotetext{gemeint sind: $date$, $month$, $week$, $time$, $datetime-local$, $number$, $range$}
Die ersten beiden begrenzen die Werte auf einen Bereich oder ein Intervall.
Mit $step$ lässt sich noch die Grösse eines Schrittes einstellen.
Die textlichen Felder\footnotemark besitzen die Attribute $maxlength$, $minlength$, $pattern$ und $size$.
\footnotetext{in diesem Zusammenhang: $text$, $search$, $url$, $tel$, $email$, $password$}
Diese Texttypen zusammen mit $number$ können durch $placeholder$ einen Patzhaltertext erhalten.
Die Min- und Max-Länge schränken die Textlänge der Eingabe ein.
Um eine Eingabe weiter zu begrenzen, kann ein Muster mit Regex im $pattern$-Attribut vorgegeben werden.
Dieses wird vor dem Senden des Fodmulars validiert und bei Fehlern mit invalide markiert.
Gewisse Typen - wie z.B. $url$, $tel$ und $email$ - haben bereits eine Pattern hinterlegt.
Mit $size$ lässt sich die Grösse bzw. Anzahl sichtbarer Zeichen angeben.
Weitere Informationen zum Styling eines Input-Feldes stehen im Unterkapitel \textbf{CSS für Input-Felder}.

Das Input enhält keinen Inhalt und ist ein selbst-schliessendes Tag.
Die ARIA-Rolle ist vom Typ abhängig und kann als Textbox, Combobox, Spinbutton, Sliter, Searchbox, Telbox oder keiner spezeillen Rolle zugeordent sein.
In den hier erwähnten Möglichkeiten der Kompontente existiert eine grossflächige Browserunterstützung.
Die einzigen Ausnahmen beziehen sich auf die Typen $week$ und $month$, welche im Firefox und Safari nicht funktionieren.

\subsubsection{{\color{dgray} Input-Typen mit Datalist-Unterstützung}}
Angefangen mit den textuellen Typen existieren $text$, $search$, $number$, $email$, $url$ und $tel$.
Diese Typen Erscheinen als mehr oder weniger normales, einzeiliges Texteingabefeld.
Die ersten zwei der Auflistung verwalten einen Text ohne spezeille Anforderungen. 
Um die Werte als Zahlen zu speichern, dient der Typ $number$, welche dem UI noch zwei Buttons ergänzt.
Die letzten drei Typen sind Textfelder, welche bereits ein passendes Pattern für E-Mail, URL oder Telefonnummer hinterlegt haben.
Zudem zeigen diese Felder bei dynamischen Tastaturen eine der Situation angepasste\footnotemark.
\footnotetext{z.B. ist bei $email$ @ und . immer sichtbar, oder bei $tel$ sind die Zahlen besser dargestellt}
Typen in der Kategorie Date-Time sind $month$, $week$, $date$, $time$ und $datetime-local$.

Bei einer ungefähren Eingabe einer Zahl auf einem Intervall bietet sich $range$ an.
Im UI zeigt sich diese Komponente als Slider, wobei sich der Standard Wert in der Mitte findet.
Zur Festlegun der Limiten dienen $min$ und $max$.
Für eine Farbauswahl kann der Typ $color$ zur Anwendung kommen. 
Dieser zeigt in den gängigen Browser nach dem aufklappen ein Palette, welche sich aber in der Darstellung unterscheiden kann. 
Firefox beispielsweise greift auf die vom System gestellt Farbpalette zurück, wo Chromium-basierte Browser eine eigene bieten.

\subsubsection{{\color{dgray} CSS für Input-Felder}}
Bei einem Input bestehen mehr Möglichkeiten dieses umzugestalten.
Es existieren eigene CSS-Pseudo-Klassen, wobei folgend nur ein Teil aufgezählt wird.

\begin{itemize}
    \item \textbf{:enabled} bzw. \textbf{:disabled} - reagiert auf das $disabled$-Attribut
    \item \textbf{:read-write} bzw. \textbf{:read-only} - reagiert auf das $readonly$-Attribut
    \item \textbf{:valid} bzw. \textbf{:invalid} - reagiert auf die Client-Side Validität\footnotemark des Felds, sobald das Formular versendet wird
    \item \textbf{:user-invalid} - reagiert auf die Client-Side Validität des Felds, sobald das Feld verlassen wird
    \item \textbf{:in-range} bzw. \textbf{:out-of-range} - reagiert auf das $min$- und $max$-Attribut
    \item \textbf{:optional} bzw. \textbf{:required} - reagiert auf das $required$-Attribut
    \item \textbf{:blank} - reagiert wenn ein Feld leer ist
\end{itemize}
\footnotetext{Einhaltung der Regeln für das Input wie ein gegebenes $pattern$}

Weiter gibt es noch die das Pseudo-Element \textbf{::placeholder}, welches das Stylen des Platzhalter erlaubt.
In Feldern mit einem Cursor\footnotemark kann dieser mit dem CSS-Property \textbf{caret-color} umgefärbt werden.
\footnotetext{hier gemeint: blinkender Strich in einem Eingabefeld}
Generell kann das normale Input-Feld relativ gut designt werden - im Vergleich zur $datalist$ und dem $select$.

\clearpage
\section{Browser-Inkonsistenzen}
Die Komponenten Select \& Datalist können sich im UI und der Bedingung leicht variieren,
besonders hinsichtlich der visuellen Darstellung und des Verhaltens von Navigationspfeilen oder anderen Steuerelementen.


\subsection{UI Unterschiede}
Das $select$ ist in der geschlossenen Form immer mit mindestens einem Pfeil nach unten auf der rechten Seite ausgestattet.
Safari (OSX und iOS) verwendet sogar einen Doppelpfeil (nach oben und unten) als Icon.
Firefox zeigt das Feld in grau, während alle anderen Browser einen weissen Hintergrund verwenden.
Safari auf iOS stellt die Komponente ohne Rahmen dar und desswegen ist der Hintergrund in einem hellen Grau gehalten.
Die geöffnete Liste ist bei Firefox als einziges ein wenig konsistenter, diese in einem mittel- bis hellgrauen Container kommt.
Die anderen Browser sind einstellungsabhängig und zeigen den Container in einem dunklen grau oder weiss an.
Je nach Anzahl der enthaltenen Elemente erscheint die Liste darunter (darüber) oder überdeckt das $select$ selbst sogar.
Grunsätzlich lässt sich das Element in Safari am wenigsten stylen.

Die $datalist$ ist inkonsistenter, da bespielsweise nicht in jedem Browser ein Icon angezeigt wird.
Safari (Mac) und Firefox stellen keinen visuellen Hinweis auf einen Listen-Container dar.
Safari auf iOS zeigt hingegen in jedem Fall - bezogen auf den textuellen Typ - einen Pfeil nach unten an.
Die anderen Browser liegen dazwischen und blenden das Icon (Dreieck nach unten) auf der rechten Seite beim Hovern oder beim Besitzen des Fokus ein.
Das Öffnen der Liste in Firefox unterscheidet sich ebenfalls, da, wenn das Feld den Fokus noch nicht besitzt, zwei Mal geklickt werden muss.
Die anderen Browser lassen die Liste bereits beim ersten Kick erscheinen.
Die Liste selbst verhält sich je nach Browser und Inhalt verschieden, indem sie seitlich oder darunter (darüber) erscheint.
Der Container übernimmt die Farbe des Systems nicht sowie das $select$.
Das Feld selbst wird nie überdeckt. 

\clearpage
\subsection{Edge Browser}
\import{../tables}{c.edge.tex}
Auf Windows wehält sich Edge sehr ähnlich wie Chrome, da die Codebasis beider Browser Chromium ist.
Im Design gibt es jedoch Unterschiede bei der Darstellung von den Listen oder Navigationspfeilen. 

\clearpage
\subsection{Chrome Browser}
\import{../tables}{c.chrome.tex}

Auf dem Mac verhält sich Chrome ähnlich wie auf Windows, jedoch können sich Designaspekte unterscheiden. 
Zum Beispiel könnte der Navigationspfeil in einem anderen Stil dargestellt werden oder 
die Animationen beim Öffnen von Dropdowns könnten glatter sein.

\clearpage
\subsection{Firefox Browser}
\import{../tables}{c.firefox.tex}

Firefox zeigt auf dem Mac konsistentes Verhalten wie auf Windows, jedoch mit typischen MacOS Designanpassungen. 
Elemente wie Buttons und Listen könnten abgerundetere Ecken haben und 
die Interaktions-Feedbacks können sich leicht von der Windows-Version unterscheiden.

\clearpage
\subsection{Safari Browser}
\import{../tables}{c.safari.tex}

Ähnlich wie in Chrome und Firefox auf Windows, aber Designelemente wie Schatten oder Ränder können subtiler sein. 
Safari tendiert dazu, ein minimalistischeres Design zu verwenden, was sich in der Darstellung von Dropdowns und Comboboxen widerspiegelt.

\clearpage
\subsection{Safari Browser auf iOS}
\import{../tables}{c.safari.ios.tex}


\clearpage
\subsection{Firefox \& DuckduckGo auf Android}

Auf Mobile 

\import{../tables}{c.firefox.android.tex}
\import{../tables}{c.duckduck.android.tex}


\clearpage
\subsection{Fazit}

% todo rewrite with focus on differences visible in tables
Generell ändern sich die Verhaltensweisen der Standardkomponenten in verschiedenen Browsern in Bezug auf Design und Interaktionsmechanismen. 
Hierbei spielen das Betriebssystemspezifische Styling und die individuellen Implementierungen der Browser eine Rolle. Die Grundfunktionen zeigen sich ähnlich, 
wobei etwaige Unterschiede oft in der visuellen Darstellung und in Details der Interaktion (z.B. beim Öffnen der Dropdowns oder der Navigation innerhalb der Listen) liegen. 
Wichtig ist somit die Konsistenzprüfung, um sicherzustellen, dass Webanwendungen über verschiedene Plattformen und Browser hinweg einheitlich funktionieren und eine optimale Benutzererfahrung bieten. 
Das Verhalten kann je nach Browserversion und Betriebssystem variieren. Die Analyse der Tabellen unterstreicht, dass trotz grundlegender Übereinstimmungen in der Funktionsweise von Komponenten wie Datalist und Select 
die optische Darstellung und bestimmte Interaktionsdetails je nach Betriebssystem und Browser variieren. 
Insbesondere sind solche Unterschiede in der Bedienung mit Tastaturkürzeln wie Command/Ctrl und der Visualisierung von Navigationspfeilen oder Scroll-Verhaltensweisen sichtbar. 
Diese Erkenntnisse fordern eine sorgfältige Abstimmung und eventuelle Anpassungen im Designprozess, um eine einheitliche Benutzererfahrung sicherzustellen.


\section{Anwendungsfälle}
\begin{itemize}
    \item Länderauswahl (nach Kontinenten filtern)
    \item Jahrgang auswählen (nach Dekade filtern)
\end{itemize}
