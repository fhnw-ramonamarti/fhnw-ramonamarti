\chapter{Existierende Auswahlkomponenten}
\label{chap:existingComponents}


Hier werden die bereits vorhandenen Möglichkeiten zur Darstellung einer Optionsauswahl einander gegenübergestellt.
Einer Auflistung der möglichen Funktionalitäten zeigt die Grenzen dieser Elemente auf.
Dabei spielen die UI- als auch die Interaktions-Unterschiede diese Komponenten in verschiedenen Browsern eine tragende Rolle.

Zu den bereits existierenden Komponenten zum Start dieser Arbeit zählen Buttons bzw. Links, \texttt{select}-, \texttt{datalist}-Element und das \texttt{Country Select}.
Unter dem Country Select ist das Resultat des Projekt 5 zu verstehen.
Die folgende Tabelle \ref{table:generalComparing} zeigt einen Vergleich der genannten Möglichkeiten.

\import{../tables}{c.choice.general.tex}


\section{HTML Datalist vs. Select}
\label{sec:datalistSelect}

Die folgenden Unterkapitel zeigen die Möglichkeiten, welche die HTML-Elemente \texttt{input}, \texttt{option}, \texttt{datalist} und \texttt{select} bieten.
Zudem zeigen tabellarische Gegenüberstellungen die Unterschiede und Inkonsistenzen dieser in verschiedenen Browsern und Betriebssystemen auf.
Hierbei liegt der Fokus mehr auf der Interaktion mit den Komponenten als auf der Darstellung.


\subsection{Option}
\label{sec:option}

\begin{lstlisting}[style = htmlcssjs, caption = Option Example, label = code:OptionExample]
<option value="ValueToSend">LabelToShow</option>
<option value="DefaultValue" selected>DefaultLabel</option>
<option value="Value" disabled>DisabledLabel</option>
\end{lstlisting}

Das $option$-Tag wird bei beiden Auswahlkomponenten - $datalist$ und $select$ - verwendet. 
Die einzelnen Optionen wie in Code \ref{code:OptionExample} Zeile 1 besitzen ein $value$-Attribut und einen Inhalt - das Label.
Alternativ kann das $label$-Attribut genutzt werden und der Value an die Stelle des Inhalts platziert werden.
Diese Variante wird jedoch von Firefox nicht unterstützt.
Werden $value$ und $label$ weggelassen, erhalten beide den Wert innerhalb des Tags.
Das HTML-Element kann das Attribut $disabled$ oder $selected$ erhalten.
Eine $selected$ Option wie in Code \ref{code:OptionExample} Zeile 2 wird initial ausgefüllt und funktioniert nur im Zusammenhang mit dem $select$-Container. 
Bei einem $disabled$ Element - wie in Code \ref{code:OptionExample} Zeile 3 - erscheint das UI ausgegraut.
Dieser Eintrag lässt keine Interaktion zu.

(\cite{optionMdn}) Das Designen von $option$s beschränkt sich auf die Text- / Hintergrundfarbe, welche jedoch nur in Firefox funktionieren. 
Die anderen Browser lassen kein Styling der Auswahl-Elemente zu bzw. zeigen diese nicht an.
Es ist nur ein skalarer Text als Inhalt erlaubt. 
Das bedeutet, es können keine weiteren Tags verschachtelt werden.
Als Eltern-Elemente sind $select$, $optgroup$ und $datalist$ erlaubt.
Alle gängigen Browser unterstützen dieses Tag.



\subsection{Select}
\label{sec:select}

\begin{lstlisting}[style = htmlcssjs, caption = Select Example, label = code:SelectExample]
<select name="animal">
    <option value="" disabled>Animal</option>
    <option value="dog">Dog</option>
    <option value="cat">Cat</option>
    <option value="mouse">Mouse</option>
</select>
\end{lstlisting}

Um den Wert zu speichern, benötigt das $select$-Tag ausser $option$s keine weiteren Elemente – ersichtlich im Code \ref{code:SelectExample}.
Die typischen Attribute für Eingaben-Felder wie $disabled$, $form$, $name$ und $required$ sind auch bei einem Select verwendbar.
Durch das $disabled$ zeigt sich das Element in einer ausgegrauten Erscheinung und bietet dem Nutzer keine Interaktionsmöglichkeiten mehr.
Dieses Attribut kann vom Container - z.B. $fieldset$ - geerbt werden.
Die $form$-Eigenschaft definiert das dazugehörige Formular und $name$ den Key für das Key-Value Paar bzw. Name des Feldes im Formular. 
Mit dem $required$ wird eine Eingabe erzwungen, um das Senden des Formulars freizuschalten.
Um dem Select für eine bessere Accessibility ein Label zuweisen zu können, ist eine $id$ zu vergeben, 
Standardmässig ist für dieses Feld nur eine einzelne Auswahl möglich.
Durch die Ergänzung des Attributs $multiple$ ist es möglich mehrere Werte zu markieren.
Bei einer Vorauswahl mehrerer Listenelemente durch $selected$ muss die Komponente ein Multiselect sein.
Wenn jedoch bei einem Signle-Select mehrere Optionen das $selected$-Attribut enthalten, wird das Letzte als Default eingesetzt und die Vorherigen ignoriert.
Das $autocomplete$ funktioniert wie bei anderen Inputs, indem Vorschläge vom User-Agent-Feature auftauchen.
Durch die $autofocus$-Eigenschaft sind Interaktionen mit dem Feld direkt nach dem Laden möglich.
Die Definition des $size$-Attributs steuert die Anzahl sichtbarer Elemente, wobei der Default für Single-Selects bei eins und mehrfache Auswahl bei vier liegt.
Ausser von dem $size$-Attribut erhält das Select von allen anderen in allen Browsern Unterstützung.
Die Grösse wird von den meisten Mobile-Browsern nicht supportet.

Abgesehen von der Grösse ist die Umgestaltung des Elements browserunabhängig kaum möglich.
Es gibt jedoch aufwendige Wege, den Inhalt zu klonen und durch Wrapper neu zu stylen.
In diesem Fall müssen aber die Logik und die Interaktionen für die Accessibility neu implementiert werden.
Gewisse Stylings können getroffen werden, wobei aber nicht alle Browser diese in der selben Weise übernehmen.
Durch die komplexe Struktur des Selects ist eine eigene Darstellung schwierig zu kontrollieren.

(\cite{selectMdn}) Der Inhalt des Tags können $option$s oder $optgroup$s sein.
Mehr zur $optgroup$ kommt im nachfolgenden Unterkapitel.
In der ARIA-Rolle wird die Komponente als Combobox (ohne $size$ \& $multiple$) oder Listbox gehandhabt.

Die mehrfache Auswahl rein per Tastatur bietet in der Interaktion geringere Möglichkeiten als mit der Maus.
Mit gedrückter Cmd (Mac) bzw. Ctrl (Windows) Taste können mit der Maus weitere Elemente mitausgewählt werden.
Mit dem Drücken der Shift-Taste lassen sich alle Elemente zwischen der ersten und der letzten Option markieren.
Damit sich die zuvor markierte Auswahl nicht aufhebt, ist darauf zu achten, beim Anwenden beider Techniken den Bereich mit Shift zuerst auszuwählen.
Um nur mit der Tastatur mehrere Werte zu markieren, ist es unabdingbar, zuerst zum ersten Element zu navigieren.
Mit Shift und den Pfeiltasten hoch und runter ist ein Bereich wählbar.
Firefox unterstützt noch die einzelne Mehrfachauswahl über die Tastatur, indem mit Ctrl der Fokus hoch und runter bewegt wird und mit der Leertaste selektiert werden kann.
Auf dem Mac sind die Tastenkombinationen Systembefehle.
Daher gibt es keine Alternative als sie zuvor auszuschalten bzw. umszustellen.


\subsubsection{\color{dgray} Optgroup}
\label{sec:optgroup}

\begin{lstlisting}[style = htmlcssjs, caption = Optgroup Example, label = code:OptgroupExample]
<select name="animal">
    <optgroup label="Big animal">
        <option value="dog">Dog</option>
        <option value="cat">Cat</option>
    </optgroup>
    <optgroup label="Small animal">
        <option value="mouse">Mouse</option>
        <option value="hamster">Hamster</option>
    </optgroup>
</select>
\end{lstlisting}

Die $optgroup$ dient als Zusatz-Element um in $select$s Optionen zu Gruppieren und mit einem Zwischentitel zu versehen.
Der Titel lässt sich durch das $label$-Attribut setzen, ist aber nicht selektierbar. 
Um ein ganzer Block von Auswahlmöglichkeiten auszugrauen, bietet sich das $disabled$ an.
Die Optionen, welche in diesem Tag enthalten sind, erben das Attribut.
Als Inhalt dienen $option$-Elemente und selbst besitzt es als Eltern-Tag ein $select$.
Die $optgroup$ besitzt die ARIA-Rolle einer Gruppe und wird von allen Browsern unterstützt.

% todo continue here
\subsection{Datalist}
\label{sec:datalist}

\begin{lstlisting}[style = htmlcssjs, caption = Datalist Example, label = code:DatalistExample]
<input type="text" name="animal" 
       list="data" placeholder="Animal" />
<datalist id="data">
    <option value="Dog"></option>
    <option value="Cat"></option>
    <option value="Mouse"></option>
</datalist>
\end{lstlisting}

Wie in Code \ref{code:DatalistExample} ersichtlich besteht die Datalist aus zwei Teilen - einem Input-Feld und einem Daten-Container. 
Im folgenden Unterkapitel \textbf{Input} wird detaillierter auf das Eingabe-Feld und dessen Typen eingegangen.
Die Datenliste besitzt keine speziellen eigenen Attribute.
Von den globalen Attributen sollte jedoch zumindest eine $id$ vergeben werden.
Die Id dient zur Verknüpfung der Liste mit dem Input-Feld. 

(\cite{datalistMdn}) Das Stylen der $datalist$ ist sehr begrenzt bzw. nicht möglich. 
Der Daten-Container reagiert nicht auf den Zoom des Browsers.
Gewisse Screenreader ignorieren die Vorschlagsliste und lesen diese somit nicht vor.
In der ARIA-Rolle wird das Element als Listbox interpretiert.

Die $option$s einer $datalist$ besitzen normalerweise nur ein $value$-Attribut und kein Label oder nur Inhalt.
Bei einer zusätzlichen Label-Definition kann das je nach Browser zu abweichenden Darstellungen kommen. 
Während Firefox nur das Label in der Liste anzeigt, werden in den anderen Browsern Label und Value gemeinsam visualisiert. 
Die Browser, welche Label und Value anzeigen, heben den Wert ein wenig hervor.
Die Darstellung kann zu Missverständnissen führen, da bei der Auswahl nur der Value in das Eingabe-Feld eingefügt wird. 
Die $option$s-Werte können sich dem Typ des Inputs anpassen. 
Generell unterstützen alle Browser die $datalist$, aber Firefox nur begrenzt.

(\cite{datalistMdn}) Nicht alle Browser unterstützen die Datenliste für jeden Eingabe-Typ.
Der textuelle Typ funktioniert in allen Browsern und die Liste öffnet sich nach einem Klick bzw. Doppelklick auf das Feld.
Vordefinierte Datum- und Zeit-Typen funktionieren nur in den Chromium basierten Browsern. 
Firefox und Safari zeigen den normalen Eingabe-Container, als ob die Liste nicht verknüpft ist.
Wenn eine Liste mit einer Range verwendet wird, zeigen alle Browser die Optionen durch Markierungen an der jeweiligen Stelle auf dem Slider.
Die Kombination einer $datalist$ mit einer Farbpalette zeigt eine breite, aber unterschiedliche Unterstützung. 
Unter den gängigen Browsern ist Firefox auf OSX der Einzige, welche die Liste nicht darstellt.
Zusammengefasst bieten für Listen mit den unterschliedlichsten Typen Chromium-basierte Browser die beste Unterstützung.


\subsubsection{\color{dgray} Input}
\label{sec:input}

Dieser Abschnitt behandelt nur den Teil, welcher im Bezug auf die $datalist$ von Belang ist.
Das Input - in Code \ref{code:DatalistExample} auf Zeile 1 - wird durch das $list$-Attribut mit den Auswahloptionen verknüpft.
Bei der Verwendung zusammen mit der $datalist$ ist dieses Attribut Pflicht.
Dadurch erscheinen die Auswahloptionen während der Bedienung des Input-Feldes. 

Die in diesem Kontext für alle Typen geltenden Attribute werden in diesem Paragraphen genauer behandelt.
Das $autocomplete$-Attribut dient zur Anzeige der Hinweise für das Autofill-Feature der Browser.
Wie in den früheren Abschnitten bereits beschrieben kann durch das $disabled$ die Interaktion ausgeschaltet werden und somit die Komponente deaktiviert werden.
Wenn das Eingabe-Feld ausserhalb eines Formulars platziert ist, kann durch das Attribut $form$ eine Verknüpfung zu einem auf der Seite existierenden Formular hergestellt werden.
Die Eigenschaft $name$ dient zur Identifizierung des Wertes im $value$-Attribut im abgesendeten Formular.
Der Typ des Eingabefeldes - angegeben durch das $type$ - definiert, welche UI-Erscheinung das Input im Browser erhält und welche Werte zulässig sind.
Als Standard-Typ ist $text$ definiert, wodurch dieses Attribut optional ist.
Mehr zu den für diese Arbeit wichtigen Typen ist im Unterkapitel \textbf{Input-Typen mit Datalist-Unterstützung} zu lesen.
Die $id$ als globale Eigenschaft dient bei den Eingabe-Feldern zusätzlich noch zur Verknüpfung mit einem Label-Element und somit einer bessern Accessibility.

Weiter existieren noch die zwei Attribute $readonly$ und $required$, welche bei allen Typen ausser $range$ und $color$ definiert sind.
Durch die Ergänzung der ersten Eigenschaft ist der Wert nicht mehr änderbar, aber es bleibt im Verlauf der fokussierbaren Komponenten enthalten.
Um die Eingabe zu erzwingen, kann das $required$ angewendet werden.
Die Eigenschaften $min$, $max$ und $step$ helfen numerischen Typen\footnotemark zu konfigurieren.
\footnotetext{gemeint sind: $date$, $month$, $week$, $time$, $datetime-local$, $number$, $range$}
Die ersten beiden begrenzen die Werte auf einen Bereich oder ein Intervall.
Mit $step$ lässt sich noch die Grösse eines Schrittes einstellen.
Die textlichen Felder\footnotemark besitzen die Attribute $maxlength$, $minlength$, $pattern$ und $size$.
\footnotetext{in diesem Zusammenhang: $text$, $search$, $url$, $tel$, $email$, $password$}
Diese Texttypen zusammen mit $number$ können durch $placeholder$ einen Patzhaltertext erhalten.
Die Min- und Max-Länge schränken die Textlänge der Eingabe ein.
Um eine Eingabe weiter zu begrenzen, kann ein Muster mit Regex im $pattern$-Attribut vorgegeben werden. 
Dieses wird vor dem Senden des Formulars validiert und bei Fehlern mit invalide markiert.
Gewisse Typen - wie z.B. $url$, $tel$ und $email$ - haben bereits eine Pattern hinterlegt.
Mit $size$ lässt sich die Grösse bzw. Anzahl sichtbarer Zeichen angeben.
Weitere Informationen zum Styling eines Input-Feldes stehen im Unterkapitel \textbf{CSS für Input-Felder}.

Das Input enthält keinen Inhalt und ist ein selbst-schliessendes Tag.
Die ARIA-Rolle ist vom Typ abhängig und kann als Textbox, Combobox, Spinbutton, Slider, Searchbox, Telbox oder keiner speziellen Rolle zugeordent sein.
In den hier erwähnten Möglichkeiten der Kompontente existiert eine grossflächige Browserunterstützung.
Die einzigen Ausnahmen beziehen sich auf die Typen $week$ und $month$, welche im Firefox und Safari nicht funktionieren.


\subsubsection{{\color{dgray} Input-Typen mit Datalist-Unterstützung}}
\label{sec:inutTypesDatalist}

(\cite{datalistMdn}) Angefangen mit den textuellen Typen existieren $text$, $search$, $number$, $email$, $url$ und $tel$.
Diese Typen erscheinen als mehr oder weniger normales, einzeiliges Texteingabefeld.
Die ersten zwei der Auflistung verwalten einen Text ohne spezielle Anforderungen. 
Um die Werte als Zahlen zu speichern, dient der Typ $number$, welche dem UI noch zwei Buttons ergänzt.
Die letzten drei Typen sind Textfelder, welche bereits ein passendes Pattern für E-Mail, URL oder Telefonnummer hinterlegt haben.
Zudem zeigen diese Felder bei dynamischen Tastaturen eine der Situation angepasstes Layout\footnotemark.
\footnotetext{z.B.: @ und . ist bei $email$ immer sichtbar, oder bei $tel$ sind die Zahlen besser dargestellt}
Typen in der Kategorie Date-Time sind $month$, $week$, $date$, $time$ und $datetime-local$.

Bei einer ungefähren Eingabe einer Zahl auf einem Intervall bietet sich $range$ an.
Im UI zeigt sich diese Komponente als Slider, wobei sich der Standard Wert in der Mitte findet.
Zur Festlegung der Limiten dienen $min$ und $max$.
Für eine Farbauswahl kann der Typ $color$ zur Anwendung kommen. 
Die gängigen Browser zeigen nach dem Aufklappen Farbpaletten, welche sich aber in der Darstellung unterscheiden. 
Firefox beispielsweise greift auf die vom System gestellte Farbpalette zurück, wo Chromium-basierte Browser eine Eigene bieten.


\subsubsection{{\color{dgray} CSS für Input-Felder}}
\label{sec:inputCss}

(\cite{inputMdn}) Bei einem Input bestehen mehrere Möglichkeiten dieses umzugestalten.
Es existieren eigene CSS-Pseudo-Klassen, wobei folgend nur ein Teil aufgezählt wird.

\begin{itemize}
    \item \textbf{:enabled} bzw. \textbf{:disabled} - reagiert auf das $disabled$-Attribut
    \item \textbf{:read-write} bzw. \textbf{:read-only} - reagiert auf das $readonly$-Attribut
    \item \textbf{:valid} bzw. \textbf{:invalid} - reagiert auf die Client-Side Validität\footnotemark des Felds, sobald das Formular versendet wird
    \item \textbf{:user-invalid} - reagiert auf die Client-Side Validität des Felds, sobald das Feld verlassen wird
    \item \textbf{:in-range} bzw. \textbf{:out-of-range} - reagiert auf das $min$- und $max$-Attribut
    \item \textbf{:optional} bzw. \textbf{:required} - reagiert auf das $required$-Attribut
    \item \textbf{:blank} - reagiert wenn ein Feld leer ist
\end{itemize}
\footnotetext{Einhaltung der Regeln für das Input - z.B. ein gegebenes $pattern$}

Weiter gibt es noch das Pseudo-Element \textbf{::placeholder}, welches das Stylen des Platzhalters erlaubt.
In Feldern mit einem Cursor\footnotemark kann dieser mit dem CSS-Property \textbf{caret-color} umgefärbt werden.
\footnotetext{blinkender Strich in einem Eingabefeld}
Generell kann das normale Input-Feld relativ gut designed werden - im Vergleich zur $datalist$ und dem $select$.


\clearpage
\section{Browser-Inkonsistenzen} % todo vllt anpassen abschnitt
\label{sec:browserInconsistency}

Die Komponenten Select \& Datalist variieren im UI und in der Bedienung,
besonders hinsichtlich der visuellen Darstellung und des Verhaltens von Navigationspfeilen oder anderen Steuerelementen.


\subsection{UI Unterschiede}
\label{sec:uiDifferences}

Das Select ist in der geschlossenen Form immer mit mindestens einem Pfeil nach unten auf der rechten Seite ausgestattet.
Safari (OSX und iOS) verwendet als Icon einen Doppelpfeil (nach oben und unten).
Firefox zeigt das Feld in grau, während alle anderen Browser einen weissen Hintergrund verwenden.
Safari auf iOS stellt die Komponente ohne Rahmen dar und deswegen ist der Hintergrund in einem hellen Grau gehalten.
Die geöffnete Liste ist bei Firefox als einziges relativ konsistent. 
Sie erscheint in einem mittel- bis hellgrauen Container.
Die anderen Browser sind einstellungsabhängig und zeigen den Container weiss oder dunklen grau an.
Je nach Anzahl der enthaltenen Elemente erscheint die Liste darunter (darüber) oder überdeckt das Select. % todo anders folmulieren select zu generell
Am Wenigsten lässt sich das Element in Safari stylen.

Nicht in jedem Browser wird ein Icon angezeigt, weswegen die Datalist weniger konsistent ist.
Safari (Mac) und Firefox stellen keinen visuellen Hinweis auf einen Listen-Container dar.
Safari auf iOS zeigt hingegen in jedem Fall - bezogen auf den textuellen Typ - einen nach unten zeigenden Pfeil an.
Andere Browser blenden auf der rechten Seite beim Hovern oder beim Besitzen des Fokus das Icon (Dreieck nach unten zeigend) ein.
Bei Firefox unterscheidet sich das Öffnen der Liste ebenfalls. 
Wenn das Feld den Fokus noch nicht besitzt, muss es zwei Mal angeklickt werden.
Die anderen Browser lassen die Liste bereits beim ersten Kick erscheinen.
Die Liste selbst verhält sich je nach Browser und Inhalt verschieden, indem sie seitlich oder darunter (darüber) erscheint.
Ob der Dark- bzw. Light-Mode als Container-Farbe übernommen wird, ist vom Browser und dem Anwendungskontext abhängig.
Das Feld selbst wird nie von der Liste überdeckt. 

Ein Blick auf die mobilen Browser wie iOS-Safari, Android-Firefox und -DuckDuckGo zeigen weitere UI-Unterschiede.
Im Vergleich zu den Desktop-Version zeigen Safari und Firefox das geöffnete Select für eine besser Bedienbarkeit komplett ander an.
% todo 


\clearpage
\subsection{Edge Browser}
\label{sec:edgeBrowser}

\import{../tables}{c.edge.tex}

Auf Windows verhält sich Edge sehr ähnlich wie Chrome, da die Codebasis beider Browser Chromium ist.


\clearpage
\subsection{Chrome Browser}
\label{sec:chromeBrowser}

\import{../tables}{c.chrome.tex}

Auf dem Mac verhält sich Chrome ähnlich wie auf Windows, jedoch können sich Designaspekte unterscheiden. 


\clearpage
\subsection{Firefox Browser}
\label{sec:firefoxBrowser}

\import{../tables}{c.firefox.tex}

Wie auch auf Windows zeigt Firefox auf dem Mac konsistentes Verhalten, jedoch mit typischen OSX Designanpassungen. 
Die Interaktions-Feedbacks auf Mac können sich leicht von der Windows-Version unterscheiden.


\clearpage
\subsection{Safari Browser}
\label{sec:safariBrowser}

\import{../tables}{c.safari.tex}

Safari tendiert dazu, ein minimalistisches Design zu verwenden, was sich in der Darstellung von Dropdowns und Comboboxen widerspiegelt.
Dieser Browser verwendet Designelemente wie Schatten und Ränder in anderer Form als Chrome und Firefox.
Das Select zeigt die grössten Unterschiede im UI.


\clearpage
\subsection{Browser auf Android \& iOS}
\label{sec:mobileBrowser}

\import{../tables}{c.firefox.android.tex}
\import{../tables}{c.duckduck.android.tex}

Der Vergleich der Tabellen \ref{table:interactionFirefoxAndroid} und \ref{table:interactionDuckduckAndroid} zeigt, dass auf dem selben Android-Gerät Abweichungen existieren.
Nicht alle Bedienungsmöglichkeiten führen zur selblen Reaktion.

\clearpage
\import{../tables}{c.safari.ios.tex}

Wie in den Tabellen \ref{table:interactionFirefoxAndroid} bis \ref{table:interactionSafariIos} zu sehen ist, erlauben Mobilgeräten weniger Interaktionen als Desktop-Computer.
Zum einen stellen die verschiedenen virtuellen Tastaturen eine geringere Auswahl an Tasten an.
Auf der anderen Seite öffnet sich die virtuelle Tastatur nicht in jeder Situation. 
In Bezug auf die Komponenten Select und Datalist öffnet sich die dynamische Eingabemöglichkeit nur durch das Input der Datalist.
Dies ist der Grund, wieso in den drei oben dargestellten Tabellen keine Tastatur-Interaktionen bei Selects möglich sind. 
Die Underschiede bei der Datalist entstehen dadurch, dass nicht jeder mobile Browser die Liste bei Tastatur-Eingaben offen lässt bzw. öffnet.
Dadurch erklären sich die unterscheidlichen Verhaltensweisen, welche in den Tabellen ersichtlich sind.

Die Touch-Bedienungen auf der Datalist weisen kaum eine Übereinstimmung auf. 
Das Scrollen als auch das Klicken verhalten sich bei den ausgewählten Mobile-Browsern Safari, Firefox und DuckDuckGo verschieden.
Das Select zeigt in der Interaktion mit dem Finger konsistenter. 


\clearpage
\subsection{Fazit}
\label{sec:summeryExisting}

% todo rewrite with focus on differences visible in tables
Generell ändern sich in Bezug auf Design und Interaktionsmechanismen die Verhaltensweisen der Standardkomponenten in den verschiedenen Browsern. 
Hierbei spielen das betriebssystemspezifische Styling und die individuellen Implementierungen der Browser eine Rolle. 
Die Grundfunktionen zeigen sich ähnlich, wobei etwaige Unterschiede oft in der visuellen Darstellung und in Details der Interaktion (z.B. beim Öffnen der Dropdowns oder der Navigation innerhalb der Listen) liegen. 
Um eine optimale Benutzererfahrung bieten zu können, ist die Konsistenzprüfung unabdingbar.
Es ist die Sicherstellung, dass Webanwendungen über verschiedene Plattformen und Browser hinweg einheitlich funktionieren. 
Das Verhalten kann je nach Browserversion und Betriebssystem variieren. 
Die Analyse der Tabellen unterstreicht, dass trotz grundlegender Übereinstimmungen in der Funktionsweise von Komponenten wie Datalist und Select, 
die optische Darstellung und bestimmte Interaktionsdetails je nach Betriebssystem und Browser variieren. 
Insbesondere sind solche Unterschiede in der Bedienung mit Tastaturkürzeln wie Cmd/Ctrl und der Visualisierung von Navigationspfeilen oder Scroll-Verhaltensweisen sichtbar. 
Diese Erkenntnisse fordern eine sorgfältige Abstimmung und eventuelle Anpassungen im Designprozess, um eine einheitliche Benutzererfahrung sicherzustellen.


\section{Anwendungsfälle}
\label{sec:useCases}

Die existierenden Auswahlkomponenten finden in vielen Webapplikationen Anwendung.
Das bekanneste Beispiel eines Selects die Auswahl des Geschlechts in Formularen verschiedenster Kontexte.
Diese Verwendung ist abgesehen vom Design unproblematisch.
Ein anderer Anwendungsfall ist die Auswahl eines Jahres oder Geburtstags mit drei Selects.
Das Ausfüllen dieser Situation gestaltet sich schon unangenehmer.
Dies liegt daran, dass die Suche nach dem gewünschten Wert besonders beim Jahr eher lange dauern kann.
Viele Nutzer hassen es ein Landes aus einer List von ca. 250 weltweit oder eine Ortschaft aus mehreren 100 bis 1'000 (je nach Anwendungsgebiet) auswählen zu müssen.
Dieses Fallbeispiel findet sich in vielen Alltagssitutionen - z.B. Herkunftsland, Zielort / -land oder Wohnort - wieder.
Um die Anwendung angenehmer zu gestalten, bietet sich auf den ersten Blick die Datalist mit der eingebauten Filterfunktion an.
Das Problem ist jedoch, dass in den meisten Fällen eigene Eingaben unerwünscht sind.
Alternativ bietet sich das Select als Auswahlkomponente.
Damit zieht sich die Suche nach einem bestimmten Wert in die Länge - speziell wenn die Optionen nicht in alphabetischer Reihenfolge vorliegen.
Selbst die Zuhilfenahme der Select-Suche\footnotemark ist nicht in jeder Situation hilfreich. 
\footnotetext{Durch das Drücken der Anfangsbuchstaben des gesuchten Werte springt die Selektion zur passenden Option}
Ein weiteres Anwendungsgebiet findet sich in gewissen Webshop bei den Filter- und Sortierfunktionen wieder.
Die Komponente zeigt sich unter anderem bei der Auswahl einer Grösse, Farbe oder Kategorie.
In diesem Beispiel können viele Optionen bzw. Werte ohne klare Reihenfolge in einer Liste auftauchen.
Dies führt dazu, dass das Auffinden des gewünschten Werts eher schwer fällt.
All die oben genannten und noch weitere Fälle zeigen verschiedene Probleme auf.
Die neue Komponente ermöglicht es, die unangenehmen Situationen zu aufzulösen.
Mehr Informationen dazu ist im nachfolgenden Kapitel zu lesen.
