\begin{table}[!htb]
    \tablesettings{Vergleich Interaktion Datalist \& Select in Safari (iOS, mobile)}
    \label{table:interactionSafariIos}
    \footnotesize
    \begin{adjustbox}{max width=\textwidth}
        \begin{threeparttable}
            \begin{tabular}{ l || l | l | l }
                \trrr{\bf{Kriterium}} & \bf{Datalist} & \bf{Select}   & \bf{Multiselect} \\
                \cline{2-4}           & geschlossen   & geschlossen   & geschlossen \\
                \cline{2-4}           & offen \ccgray & offen \ccgray & offen \ccgray \\
                \hline \hline
                \trrr{Buchstaben} & Schreiben             & nicht möglich         & nicht möglich         \\
                \cline{2-4}       & nicht möglich \ccgray & nicht möglich \ccgray & nicht möglich \ccgray \\
                \hline
                \trr{Leerschlag} & Schreiben             & nicht möglich         & nicht möglich         \\
                \cline{2-4}      & nicht möglich \ccgray & nicht möglich \ccgray & nicht möglich \ccgray \\
                \hline
                \trr{Backspace} & Symbol löschen        & nicht möglich         & nicht möglich         \\
                \cline{2-4}     & nicht möglich \ccgray & nicht möglich \ccgray & nicht möglich \ccgray \\
                \hline \hline
                \trrr{Enter} & \tbbr{Formular senden / \\ ohne Form nichts} & nicht möglich         & nicht möglich         \\
                \cline{2-4}  & nicht möglich \ccgray                        & nicht möglich \ccgray & nicht möglich \ccgray \\
                \hline \hline
                \trrrr{Scroll} & Browser-Default\tnote{1}                                                                       & Browser-Default\tnote{1}                                                        & Browser-Default\tnote{1} \\
                \cline{2-4}    & \tbbr{\emph{Aussen}: Liste schliessen \\ \emph{Innen}: Werte bewegen \\ ohne Änderung} \ccgray & \tbbr{\emph{Aussen}: - \\ \emph{Innen}: Werte bewegen \\ ohne Änderung} \ccgray & \tbbr{\emph{Aussen}: Browser-Default\tnote{1} \\ \emph{Innen}: Werte bewegen \\ ohne Änderung} \ccgray \\
                \hline
                \trr{Click} & \tbbr{\emph{Pfeil}: Liste \\ gefiltert öffnen\tnote{2}} & Liste öffnen                                    & Liste öffnen \\
                \cline{2-4} & \tbbr{Wert wählen, \\ Liste schliessen} \ccgray         & \tbbr{Wert wählen, \\ Liste schliessen} \ccgray & \tbbr{\emph{Aussen}: Liste schliessen \\ \emph{Innen}: Wert wählen / \\ abwählen} \ccgray \\
                \hline \hline
                \trr{\tbbr{Schütteln\\ (Undo / Redo)}} & iOS-Default\tnote{3}         & -         & - \\
                \cline{2-4}                            & iOS-Default\tnote{3} \ccgray & - \ccgray & - \\
                \hline 
            \end{tabular}
            \begin{tablenotes}
                \scriptsize
                \item[1] Gleiche Aktion wie wenn die Komponente nicht fokussiert ist; Aktion vom Browser bzw. Betriebssystem vorgegeben
                \item[2] Filter: Leerschlag als normales Symbol gezählt; Liste verändern je nach Anzahl passender Werte
                \item[3] Rückgängig bzw. Wiederholen, wenn zuvor im Feld etwas getippt wurde; bei nur Wert wählen $\Rightarrow$ keine Aktion
            \end{tablenotes}
        \end{threeparttable}
    \end{adjustbox}
\end{table}
