\begin{table}[ht!]
    \label{table:interactionNewComponent}
    \tablesettings{Aktionen bei der neuen Komponente}
    \rowcolors{2}{white}{gray!20}
    \footnotesize
    \begin{threeparttable}
        \begin{tabular}{ l || l | l }
            \bf{Kriterium}    & \bf{geschlossen} & \bf{offen} \\
            \hline \hline
            $\uparrow$ / $\downarrow$     & Wert wechseln       & Wert wechseln       \\
            \hline
            $\leftarrow$ / $\rightarrow$  & -                   & Spalte wechseln     \\
            \hline
            Buchstaben  & Wert ändern auf Suchergebnis\tnote{1} & Wert ändern auf Suchergebnis\tnote{1}  \\
            \hline
            Leerschlag  & Liste öffnen    & Wert ändern auf Suchergebnis\tnote{1}     \\
            \hline
            Backspace   & Auswahl löschen & Auswahl löschen        \\
            \hline
            Esc         & -               & Liste schliessen       \\
            \hline \hline
            Enter       & \tbbr{Formular senden / \\ ohne Form nichts} & Wert ändern  \\
            \hline
            Tab         & Inputfeld wechseln              & Spalte wechseln     \\
            \hline
            PageUp / PageDown  & Browser-Default\tnote{2} & \tbbr{Wert an View-Start / Ende \\ dann Seitenweise wählen} \\
            \hline
            Home / End & Erster / letzter Wert wählen     & Erster / letzter Wert wählen  \\
            \hline \hline
            Scroll     & Browser-Default\tnote{2}         & \tbbr{\textit{Aussen}: Liste schliessen \\
                                                                  \textit{Innen}: Werte bewegen \\ ohne Fokus zu ändern} \\
            \hline
            Hover      & -                & Wert highlighten       \\
            \hline
            Click      & Liste öffnen     & \tbbr{\textit{in Liste}: Wert wählen \\
                                                  \textit{in Wertfeld}: Liste schliessen} \\
            \hline \hline
            Ctrl \& Z (Undo)    & -       & Undo auf gewählten Wert ausführen         \\
            \hline
            Ctrl \& Y (Redo)    & -       & Redo auf gewählten Wert ausführen         \\
            \hline
        \end{tabular}
        \begin{tablenotes}
            \scriptsize
            \item[1] Suche: Erster zu der Eingabe passender Wert aus der Liste, wenn Eingabe nicht passend $\Rightarrow$ letzter noch übereinstimmender Wert; 
                            Liste unverändert; nach Pause / Debounce-Ablauf $\Rightarrow$ neue Suche
            \item[2] Gemeint: Gleiche Aktion wie wenn die Komponente nicht fokussiert ist; Aktion vom Browser bzw. Betriebssystem vorgegeben
        \end{tablenotes}
    \end{threeparttable}
\end{table}
