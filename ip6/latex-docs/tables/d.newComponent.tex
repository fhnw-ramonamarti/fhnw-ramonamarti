\begin{table}[!htb] % todo check if not conflicted with bullets
    \tablesettings{Aktionen bei der ersten Version der neuen Komponente}
    \label{table:interactionNewComponent}
    \rowcolors{2}{white}{gray!20}
    \footnotesize
    \begin{adjustbox}{max width=\textwidth}
        \begin{threeparttable}
            \begin{tabular}{ l || l | l }
                \bf{Kriterium}    & \bf{geschlossen} & \bf{offen} \\
                \hline \hline
                $\uparrow$ / $\downarrow$     & Wert ändern       & Wert / Kategorie ändern \\
                \hline
                $\leftarrow$ / $\rightarrow$  & -                 & Spalte ändern (Cursor Position) \\
                \hline
                Buchstaben  & Wert auf Suchergebnis\tnote{1} \ ändern & \tbbr{Wert / Kategorie auf \\ Suchergebnis\tnote{1} \ ändern} \\
                \hline
                Leerschlag  & Liste öffnen    & Wert ändern       \\
                \hline
                Backspace   & Auswahl löschen & Auswahl löschen   \\
                \hline
                Esc         & -               & Liste schliessen  \\
                \hline \hline
                Enter       & -               & Wert ändern       \\
                \hline
                Tab         & Input-Feld verlassen            & \tbbr{Liste schliessen \& \\ Feld verlassen } \\
                \hline
                PageUp / PageDown  & Browser-Default\tnote{2} & \tbbr{Jeder 10te Wert wählen} \\
                \hline
                Home / End & Erster / letzter Wert wählen     & Erster / letzter Wert wählen  \\
                \hline \hline
                Scroll     & Browser-Default\tnote{2}         & \tbbr{\emph{Aussen}: Liste bleibt offen \\
                                                                    \emph{Innen}: Werte bewegen \\ mit Highlight-Änderung} \\
                \hline
                Hover      & -                & Wert highlighten \\
                \hline
                Click      & Liste öffnen     & \tbbr{\emph{in Liste}: Wert wählen \\
                                                    \emph{in Wertfeld}: Liste schliessen} \\
                \hline
            \end{tabular}
            \begin{tablenotes}
                \scriptsize
                \item[1] Suche: Erster mit dem eingaben Symbol passender Wert aus der Liste, wenn Eingabe nicht passend $\Rightarrow$ nächster nachfolgender Wert; 
                                Liste unverändert; nach jedem Symbol $\Rightarrow$ neue Suche
                \item[2] Gleiche Aktion wie wenn die Komponente nicht fokussiert ist; Aktion vom Browser bzw. Betriebssystem vorgegeben
            \end{tablenotes}
        \end{threeparttable}
    \end{adjustbox}
\end{table}
