\begin{table}[!htb] 
    \tablesettings{Aktionen bei der ersten Version der neuen Komponente}
    \label{table:interactionNewComponent}
    \rowcolors{2}{white}{gray!20}
    \footnotesize
    \begin{adjustbox}{max width=0.95\textwidth}
        \begin{threeparttable}
            \begin{tabular}{ l || l | l }
                \bf{Kriterium}    & \bf{geschlossen} & \bf{offen} \\
                \hline \hline
                $\uparrow$ / $\downarrow$    & Selektion ändern & Cursor Position ändern                 \\
                \hline
                $\leftarrow$ / $\rightarrow$ & -                & Cursor Position ändern                 \\
                \hline
                Buchstaben  & \tbbr{Selektion auf Such-\\ 
                                    ergebnis\tnote{1} \ ändern} & \tbbr{Cursor Position auf \\ 
                                                                        Suchergebnis\tnote{1} \ ändern}  \\
                \hline
                Leerschlag  & Liste öffnen                & Selektion ändern  \\
                \hline
                Backspace   & Selektion löschen           & Selektion löschen \\
                \hline
                Delete      & Selektion löschen           & Selektion löschen \\
                \hline
                Esc         & -                           & Liste schliessen  \\
                \hline \hline
                Enter       & -                           & Selektion ändern  \\
                \hline
                Tab         & Input-Feld verlassen        & \tbbr{Liste schliessen \& \\ 
                                                                  Input-Feld verlassen}                  \\
                \hline
                PageUp / PageDown & Fenster scrollen      & \tbbr{Cursor Position auf \\ 
                                                                  jeden 10. Wert ändern}                 \\
                \hline
                Home / End & \tbbr{Selektion auf ersten/ \\ 
                                   letzten Wert ändern}   & \tbbr{Cursor Position auf ersten/ \\ 
                                                                  letzten Wert ändern}                   \\
                \hline \hline
                Scroll     & Fenster scrollen             & \tbbr{\emph{Aussen}: Liste bleibt offen \\
                                                                  \emph{Innen}: Liste scrollen \& \\ 
                                                                                Highlight ändern}        \\
                \hline
                Hover      & -                            & Highlight ändern                             \\
                \hline
                Click      & Liste öffnen                 & \tbbr{\emph{in Liste}: Selektion ändern \\
                                                                  \emph{in Wertefeld}: Liste schliessen} \\
                \hline
            \end{tabular}
            \begin{tablenotes}
                \scriptsize
                \item[*] Änderung der Selektion bewirkt Änderung der Cursor Position auf den selben Wert
                \item
                \item[1] Suche: Erster mit dem eingegebenen Symbol passender Wert aus der Liste, wenn Eingabe nicht passend $\Rightarrow$ nächster nachfolgender Wert; 
                                Liste unverändert; nach jedem Symbol $\Rightarrow$ neue Suche
            \end{tablenotes}
        \end{threeparttable}
    \end{adjustbox}
\end{table}
