\begin{table}[!htb]
    \tablesettings{Aktionen bei der Länderauswahl-Komponente}
    \label{table:interactionCountryInput}
    \rowcolors{2}{white}{gray!20}
    \footnotesize
    \begin{adjustbox}{max width=0.95\textwidth}
        \begin{threeparttable}
            \begin{tabular}{ l || l | l }
                \bf{Kriterium}    & \bf{geschlossen} & \bf{offen} \\
                \hline \hline
                $\uparrow$ / $\downarrow$    & $\downarrow$ : Liste öffnen & \tbbr{\emph{Kontinent}: Selektion ändern \\ 
                                                                                   \emph{Land}: Highlight ändern}  \\
                \hline
                $\leftarrow$ / $\rightarrow$ & -                           & Highlight ändern                         \\
                \hline
                Buchstaben & \tbbr{Selektion auf Such-\\ 
                                   ergebnis\tnote{1} \ ändern}             & \tbbr{Highlight auf Such-\\ 
                                                                                   ergebnis\tnote{1} \ ändern}        \\
                \hline
                Leerschlag & Liste öffnen          & Selektion ändern  \\
                \hline
                Backspace  & Selektion löschen     & Selektion löschen \\
                \hline
                Esc        & -                     & Liste schliessen  \\
                \hline \hline
                Enter      & Liste öffnen          & Selektion ändern  \\
                \hline
                Tab        & Input-Feld verlassen  & \tbbr{Liste schliessen \& \\ 
                                                           Input-Feld verlassen }                 \\
                \hline \hline
                Scroll     & Fenster scrollen      & \tbbr{\emph{Aussen}: Liste bleibt offen \\
                                                           \emph{Innen}: Liste scrollen \& \\ 
                                                                         Highlight ändern}        \\
                \hline
                Hover      & -                     & Highlight ändern                             \\
                \hline
                Click      & Liste öffnen          & \tbbr{\emph{in Liste}: Selektion ändern \\
                                                           \emph{in Wertefeld}: Liste schliessen} \\
                \hline
            \end{tabular}
            \begin{tablenotes}
                \scriptsize
                \item[*] Änderung der Selektion bewirkt Änderung des Highlights auf den selben Wert
                \item[*] Highlight und Cursor-Position besitzen selben Wert; kein Unterschied
                \item
                \item[1] Suche: Erster mit dem eingegebenen Symbol passender Wert aus der Liste, wenn Eingabe nicht passend $\Rightarrow$ keine Aktion; 
                                Liste unverändert; nach Pause / Debounce-Ablauf $\Rightarrow$ neue Suche
            \end{tablenotes}
        \end{threeparttable}
    \end{adjustbox}
\end{table}
