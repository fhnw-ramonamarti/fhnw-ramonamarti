\begin{table}[!htb]
    \tablesettings{Aktionen bei der Länderauswahl Komponente}
    \label{table:interactionCountryInput}
    \rowcolors{2}{white}{gray!20}
    \footnotesize
    \begin{adjustbox}{max width=\textwidth}
        \begin{threeparttable}
            \begin{tabular}{ l || l | l }
                \bf{Kriterium}    & \bf{geschlossen} & \bf{offen} \\
                \hline \hline
                $\uparrow$ / $\downarrow$     & \emph{$\downarrow$}: Liste öffnen & \tbbr{\emph{Kontinent}: Wert ändern \\ \emph{Land}: Highlight ändern } \\
                \hline
                $\leftarrow$ / $\rightarrow$  & -                                 & Spalte ändern \\
                \hline
                Buchstaben  & Wert auf Suchergebnis\tnote{1} \ ändern & \tbbr{Highlight auf \\ Suchergebnis\tnote{1} \ ändern} \\
                \hline
                Leerschlag  & Liste öffnen    & Wert ändern       \\
                \hline
                Backspace   & Auswahl löschen & Auswahl löschen   \\
                \hline
                Esc         & -               & Liste schliessen  \\
                \hline \hline
                Enter       & Liste öffnen    & Wert ändern       \\
                \hline
                Tab         & Input-Feld verlassen            & \tbbr{Liste schliessen \& \\ Feld verlassen } \\
                \hline \hline
                Scroll     & Browser-Default\tnote{2}         & \tbbr{\emph{Aussen}: Liste bleibt offen \\
                                                                    \emph{Innen}: Werte bewegen \& \\ Highlight ändern} \\
                \hline
                Hover      & -                & Highlight ändern \\
                \hline
                Click      & Liste öffnen     & \tbbr{\emph{in Liste}: Wert wählen \\
                                                      \emph{in Wertefeld}: Liste schliessen} \\
                \hline
            \end{tabular}
            \begin{tablenotes}
                \scriptsize
                \item[*] Änderung der Selektion bewirkt Änderung des Highlights auf den selben Wert
                \item[*] Highlight und Cursor Position besitzen selben Wert; kein Unterschied
                \item
                \item[1] Suche: Erster mit dem eingegebenen Symbol passender Wert aus der Liste, wenn Eingabe nicht passend $\Rightarrow$ keine Aktion; 
                                Liste unverändert; nach Pause / Debounce-Ablauf $\Rightarrow$ neue Suche
                \item[2] Gleiche Aktion wie wenn die Komponente nicht fokussiert ist; Aktion vom Browser bzw. Betriebssystem vorgegeben
            \end{tablenotes}
        \end{threeparttable}
    \end{adjustbox}
\end{table}
