\begin{table}[ht!]
    \label{table:interactionFirefox}
    \tablesettings{Vergleich Interaktion Datalist \& Select in Firefox (Mac / Windows)}
    \footnotesize
    \begin{threeparttable}
        \begin{tabular}{ l || l | l | l }
            \trrr{\bf{Kriterium}} & \bf{Datalist} & \bf{Select}   & \trrr{\bf{Multiselect}} \\
            \cline{2-3}           & geschlossen   & geschlossen   &  \\
            \cline{2-3}           & offen \ccgray & offen \ccgray &  \\
            \hline \hline
            \trr{$\uparrow$ / $\downarrow$} & Liste öffnen               & Wert wechseln         & \trr{Wert wechseln} \\
            \cline{2-3}                     & Highlight wechseln \ccgray & Wert wechseln \ccgray &  \\
            \hline
            \trr{$\leftarrow$ / $\rightarrow$} & Cursor bewegen\tnote{1}                                      & Wert wechseln & \trr{Wert wechseln} \\
            \cline{2-3}                        & \tbbr{Cursor bewegen\tnote{1} / \\ Highlight wählen} \ccgray & - \ccgray     &  \\
            \hline
            \trrr{Buchstaben} & Schreiben, Liste gefiltert öffnen\tnote{2} & \tbbr{Wert ändern auf \\ Suchergebnis\tnote{3}}         & \trrr{\tbbr{Auswahl aufheben, \\ Wert ändern auf \\ Suchergebnis\tnote{3}}} \\
            \cline{2-3}       & Schreiben, Liste filtern\tnote{2} \ccgray  & \tbbr{Wert ändern auf \\ Suchergenbis\tnote{3}} \ccgray & \\
            \hline
            \trr{Leerschlag} & Schreiben, Liste gefiltert öffnen\tnote{2} & Liste öffnen & \trr{-} \\
            \cline{2-3}      & Schreiben, Liste filtern\tnote{2} \ccgray  & - \ccgray    & \\
            \hline
            \trr{Backspace} & \tbbr{Symbol löschen, Liste gefiltert\tnote{2} \\ öffnen (wenn Feld nicht leer)} & -         & \trr{-} \\
            \cline{2-3}     & Symbol löschen, Liste filtern\tnote{2} \ccgray                                   & - \ccgray & \\
            \hline
            \trr{Esc}   & -                        & -                        & \trr{-} \\
            \cline{2-3} & Liste schliessen \ccgray & Liste schliessen \ccgray & \\
            \hline \hline
            \trrr{Enter} & \tbbr{Formular senden / \\ ohne Form nichts}                     & -                                               & \trr{-} \\
            \cline{2-3}  & \tbbr{Highlight wählen / \\ ohne Highlight Form senden } \ccgray & \tbbr{Wert wählen, \\ Liste schliessen} \ccgray &  \\
            \hline
            \trr{Tab}   & Input-Feld wechseln         & Input-Feld wechseln      & \trr{Input-Feld wechseln} \\
            \cline{2-3} & Input-Feld wechseln \ccgray & Liste schliessen \ccgray & \\
            \hline
            \trrr{\tbbr{PageUp /\\ PageDown}} & Liste öffnen                                                              & Jeden 20. Wert wählen                                                & \trrr{\tbbr{Wert an nächster \\ View-Höhe wählen }} \\
            \cline{2-3}                       & \tbbr{Wert an View-Start / -Ende \\ dann Seitenweise highlighten} \ccgray & \tbbr{Wert an View-Start / -Ende \\ dann Seitenweise wählen} \ccgray & \\
            \hline
            \trrr{Home / End} & \tbbr{Wert von Listen-Anfang / \\ -Ende highlighten}         & Erster / letzter Wert wählen         & \trr{\tbbr{Erster / letzter \\ Wert wählen}} \\
            \cline{2-3}       & \tbbr{Wert von Listen-Anfang / \\ -Ende highlighten} \ccgray & Erster / letzter Wert wählen \ccgray & \\
            \hline \hline
            \trrrr{Scroll} & Browser-Default\tnote{4}                                                                                       & Browser-Default\tnote{4}                                                                                      & \trrrr{\tbbr{\textit{Innen}: Werte \\ bewegen ohne \\ Highlight / Wert \\ zu ändern}} \\
            \cline{2-3}    & \tbbr{\textit{Aussen}: Liste schliessen \\ \textit{Innen}: Werte bewegen, \\ Highlight ändert am Ende} \ccgray & \tbbr{\textit{Aussen}: Liste schliessen \\ \textit{Innen}: Werte bewegen \\ ohne Highlight zu ändern} \ccgray & \\
            \hline
            \trr{Hover} & UI-Anpassung               & -                          & \trr{-} \\
            \cline{2-3} & Highlight wechseln \ccgray & Highlight wechseln \ccgray & \\
            \hline
            \trr{Click} & Liste öffnen        & Liste öffnen                                    & \trr{\tbbr{Auswahl aufheben, \\ Wert wählen}} \\
            \cline{2-3} & Wert wählen \ccgray & \tbbr{Wert wählen, \\ Liste schliessen} \ccgray & \\
            \hline \hline
            \trr{\tbbr{Ctrl \& Z\\ (Undo)}} & Undo im Input         & -         & \trr{-} \\
            \cline{2-3}                     & Undo im Input \ccgray & - \ccgray & \\
            \hline
            \trr{\tbbr{Ctrl \& Y\\ (Redo)}} & Redo im Input         & -         & \trr{-} \\
            \cline{2-3}                     & Redo im Input \ccgray & - \ccgray & \\
            \hline
        \end{tabular}
        \begin{tablenotes}
            \scriptsize
            \item[1] Hier: blinkender Strich in einem Eingabefeld
            \item[2] Filter: Leerschlag als normales Symbol gezählt; 
                            Liste verändern je nach Anzahl passender Werte
            \item[3] Suche: Erster zu der Eingabe passender Wert aus der Liste, wenn Eingabe nicht passend $\Rightarrow$ letzter noch übereinstimmender Wert; 
                            Liste unverändert; nach Pause / Debounce-Ablauf $\Rightarrow$ neue Suche
            \item[4] Gemeint: Gleiche Aktion wie wenn die Komponente nicht fokussiert ist; Aktion vom Browser bzw. Betriebssystem vorgegeben
        \end{tablenotes}
    \end{threeparttable}
\end{table}
