\renewcommand{\colwidth}{0.24\textwidth} 
\begin{table}[!htb]
    \tablesettings{Vergleich Interaktion Datalist \& Select in Firefox (Mac / Windows)}
    \label{table:interactionFirefox}
    \footnotesize
    \begin{adjustbox}{max width=0.95\textwidth}
        \begin{threeparttable}
            \begin{tabular}{ l || l | l | l }
                                                  & \bf{Datalist} & \bf{Select}   & \bf{Multiselect} \\
                \cline{2-4}                       & geschlossen   & geschlossen   &  -               \\
                \cline{2-4} \trrr{\bf{Kriterium}} & offen \ccgray & offen \ccgray &  offen \ccgray   \\
                \hline \hline
                                                & Liste öffnen             & Selektion ändern         & \ccgray \\
                \cline{2-3}
                \trr{$\uparrow$ / $\downarrow$} & Highlight ändern \ccgray & Selektion ändern \ccgray & \trr{Selektion ändern} \ccgray \\
                \hline
                                                    & Cursor\tnote{1} \ bewegen                                     & Selektion ändern & \ccgray \\
                \cline{2-3}
                \trrr{$\leftarrow$ / $\rightarrow$} & \tbbr{\emph{in Liste}: Highlight wählen \\ 
                                                            \emph{in Wertefeld}: Cursor\tnote{1} \ bewegen} \ccgray & - \ccgray        & \trr{Selektion ändern} \ccgray \\
                \hline
                                  & Schreiben \& Liste öffnen                   & \tbbr{Selektion auf Such-\\ 
                                                                                        ergebnis\tnote{3} \ ändern}         & \ccgray \\
                \cline{2-3}
                \trrr{Buchstaben} & Schreiben \& Liste filtern\tnote{2} \ccgray & \tbbr{Selektion auf Such-\\ 
                                                                                        ergebnis\tnote{3} \ ändern} \ccgray & \trbbr{3}{\colwidth}{Selektion aufheben \& Selektion auf Suchergebnis\tnote{3} \ ändern} \ccgray \\
                \hline
                                 & Schreiben \& Liste öffnen                   & Liste öffnen & \ccgray \\
                \cline{2-3}
                \trr{Leerschlag} & Schreiben \& Liste filtern\tnote{2} \ccgray & - \ccgray    & \trr{-} \ccgray \\
                \hline
                                 & \tbbr{Löschen \& Liste öffnen \\ 
                                         (wenn Feld nicht leer)}             & -         & \ccgray \\
                \cline{2-3}
                \trrr{Backspace} & Löschen \& Liste filtern\tnote{2} \ccgray & - \ccgray & \trr{-} \ccgray \\
                \hline
                            & -                        & -                        & \ccgray \\
                \cline{2-3}
                \trr{Esc}   & Liste schliessen \ccgray & Liste schliessen \ccgray & \trr{-} \ccgray \\
                \hline \hline
                             & \tbbr{\emph{in Formular}: senden \\ 
                                     \emph{sonst}: -}                      & -                               & \ccgray \\
                \cline{2-3}
                \trrr{Enter} & \tbbr{\emph{mit Highlight}: ändern \\ 
                                     \emph{ohne}: Form senden / -} \ccgray & \tbbr{Selektion ändern \& \\ 
                                                                                   Liste schliessen} \ccgray & \trrr{-} \ccgray \\
                \hline
                            & Input-Feld verlassen         & Input-Feld verlassen     & \ccgray \\
                \cline{2-3}
                \trr{Tab}   & Input-Feld verlassen \ccgray & Liste schliessen \ccgray & \trr{Input-Feld verlassen} \ccgray \\
                \hline
                                                  & Liste öffnen                           & \tbbr{Selektion auf jeden \\ 
                                                                                                   20. Wert ändern}                 & \ccgray \\
                \cline{2-3}
                \trrr{\tbbr{PageUp /\\ PageDown}} & \tbbr{Highlight auf View-Rand \\ 
                                                          dann seitenweise ändern} \ccgray & \tbbr{Selektion auf View-Rand \\ 
                                                                                                   dann seitenweise ändern} \ccgray & \trbbr{3}{\colwidth}{Selektion auf View-Rand ändern} \ccgray \\
                \hline
                                  & \tbbr{Highlight auf ersten/ \\ 
                                          letzten Wert ändern}         & \tbbr{Selektion auf ersten/ \\ 
                                                                               letzten Wert ändern}         & \ccgray \\
                \cline{2-3}
                \trrr{Home / End} & \tbbr{Highlight auf ersten/ \\ 
                                          letzten Wert ändern} \ccgray & \tbbr{Selektion auf ersten/ \\ 
                                                                               letzten Wert ändern} \ccgray & \trbbr{3}{\colwidth}{Selektion auf ersten/ letzten Wert ändern} \ccgray \\
                \hline \hline
                             & Fenster scrollen                                      & Fenster scrollen                             & \ccgray \\
                \cline{2-3}
                \trr{Scroll} & \tbbr{\emph{Aussen}: Liste schliessen \\ 
                                     \emph{Innen}: Liste scrollen \& \\ 
                                                   Highlight ändert am Ende} \ccgray & \tbbr{\emph{Aussen}: Liste schliessen \\ 
                                                                                              \emph{Innen}: Liste scrollen} \ccgray & \trbbr{2}{\colwidth}{\emph{Aussen}: Fenster scrollen \\ 
                                                                                                                                                           \emph{Innen}: Liste scrollen} \ccgray \\
                \hline
                            & UI-Anpassung             & -                        & \ccgray \\
                \cline{2-3}
                \trr{Hover} & Highlight ändern \ccgray & Highlight ändern \ccgray & \trr{-} \ccgray \\
                \hline
                            & Liste öffnen             & Liste öffnen                    & \ccgray \\
                \cline{2-3}
                \trr{Click} & Selektion ändern \ccgray & \tbbr{Selektion ändern \& \\ 
                                                                Liste schliessen} \ccgray & \trbbr{2}{\colwidth}{Selektion aufheben \& Selektion ändern} \ccgray \\
                \hline
            \end{tabular}
            \begin{tablenotes}
                \scriptsize
                \item[1] Blinkender Strich in einem Eingabefeld
                \item[2] Filter: Leerschlag als normales Symbol gezählt; 
                                Liste verändern je nach Anzahl passender Werte
                \item[3] Suche: Erster zu der Eingabe passender Wert aus der Liste, wenn Eingabe nicht passend $\Rightarrow$ letzter noch übereinstimmender Wert; 
                                Liste unverändert; nach Pause / Debounce-Ablauf $\Rightarrow$ neue Suche
            \end{tablenotes}
        \end{threeparttable}
    \end{adjustbox}
\end{table}
