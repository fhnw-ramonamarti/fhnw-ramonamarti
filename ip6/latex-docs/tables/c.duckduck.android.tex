\begin{table}[!htb]
    \tablesettings{Vergleich Interaktion Datalist \& Select in DuckDuckGo (Android, mobile)}
    \label{table:interactionDuckduckAndroid}
    \footnotesize
    \begin{adjustbox}{max width=0.9\textwidth}
        \begin{threeparttable}
            \begin{tabular}{ l || l | l | l }
                                                  & \bf{Datalist} & \bf{Select}   & \bf{Multiselect} \\
                \cline{2-4}                       & geschlossen   & geschlossen   & geschlossen      \\
                \cline{2-4} \trrr{\bf{Kriterium}} & offen \ccgray & offen \ccgray & offen \ccgray    \\
                \hline \hline
                                 & Schreiben \& Liste öffnen                    & nicht möglich         & nicht möglich             \\
                \cline{2-4}
                \trr{Buchstaben} & Schreiben \& Liste filtern\tnote{1} \ccgray  & nicht möglich \ccgray & nicht möglich \ccgray     \\
                \hline
                                 & Schreiben \& Liste öffnen                    & nicht möglich         & nicht möglich             \\
                \cline{2-4}
                \trr{Leerschlag} & Schreiben \& Liste filtern\tnote{1} \ccgray  & nicht möglich \ccgray & nicht möglich \ccgray     \\
                \hline
                                 & \tbbr{Löschen \& Liste öffnen \\ 
                                         (wenn Feld nicht leer)}                & nicht möglich         & nicht möglich             \\
                \cline{2-4}
                \trrr{Backspace} & Löschen \& Liste filtern\tnote{1} \ccgray    & nicht möglich \ccgray & nicht möglich \ccgray     \\
                \hline \hline
                                 & \tbbr{\emph{in Formular}: senden \\ 
                                         \emph{ohne}: Input-Feld verlassen}         & nicht möglich         & nicht möglich         \\
                \cline{2-4}
                \trrr{Enter}     & \tbbr{\emph{in Formular}: senden \\ 
                                         \emph{ohne}: Input-Feld verlassen} \ccgray & nicht möglich \ccgray & nicht möglich \ccgray \\
                \hline \hline
                             & Fenster scrollen                            & Fenster scrollen                            & Fenster scrollen \\
                \cline{2-4}
                \trr{Scroll} & \tbbr{\emph{Aussen}: Liste bleibt offen \\ 
                                     \emph{Innen}: Liste scrollen} \ccgray & \tbbr{\emph{Aussen}: - \\ 
                                                                                   \emph{Innen}: Liste scrollen} \ccgray & \tbbr{\emph{Aussen}: - \\ 
                                                                                                                                 \emph{Innen}: Liste scrollen} \ccgray       \\
                \hline
                            & \tbbr{\emph{in Feld}: - \\ 
                                    \emph{Pfeil}: Liste öffnen}             & Liste öffnen                                  & Liste öffnen \\
                \cline{2-4}
                \trrr{Click} & \tbbr{\emph{Aussen}: Liste schliessen \\ 
                                     \emph{Innen}: Selektion ändern \& \\ 
                                                   Liste schliessen} \ccgray & \tbbr{\emph{Aussen}: - \\ 
                                                                                     \emph{Innen}: Selektion ändern \& \\ 
                                                                                                   Liste schliessen} \ccgray & \tbbr{\emph{Aussen}: - \\ 
                                                                                                                                     \emph{Innen}: Selektion ändern} \ccgray \\
                \hline 
            \end{tabular}
            \begin{tablenotes}
                \scriptsize
                \item[*] nicht möglich: Virtuelle Tastatur ist nicht sichtbar, somit kann keine Interaktion stattfinden
                \item[]
                \item[1] Filter: Leerschlag als normales Symbol gezählt; Liste verändern je nach Anzahl passender Werte; nicht scrollbar
            \end{tablenotes}
        \end{threeparttable}
    \end{adjustbox}
\end{table}
