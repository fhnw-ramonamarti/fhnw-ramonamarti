\begin{table}[!htb]
    \tablesettings{Vergleich Interaktion Datalist \& Select in DuckDuckGo (Android, mobile)}
    \label{table:interactionDuckduckAndroid}
    \footnotesize
    \begin{threeparttable}
        \begin{tabular}{ l || l | l | l }
            \trrr{\bf{Kriterium}} & \bf{Datalist} & \bf{Select}   & \bf{Multiselect} \\
            \cline{2-4}           & geschlossen   & geschlossen   & geschlossen \\
            \cline{2-4}           & offen \ccgray & offen \ccgray & offen \ccgray \\
            \hline \hline
            \trrr{Buchstaben} & Schreiben, Liste gefiltert öffnen\tnote{1} & nicht möglich         & nicht möglich         \\
            \cline{2-4}       & Schreiben, Liste filtern\tnote{1} \ccgray  & nicht möglich \ccgray & nicht möglich \ccgray \\
            \hline
            \trr{Leerschlag} & Schreiben, Liste gefiltert öffnen\tnote{1} & nicht möglich         & nicht möglich         \\
            \cline{2-4}      & Schreiben, Liste filtern\tnote{1} \ccgray  & nicht möglich \ccgray & nicht möglich \ccgray \\
            \hline
            \trr{Backspace} & \tbbr{Symbol löschen, Liste gefiltert\tnote{1} \\ öffnen (wenn Feld nicht leer)} & nicht möglich         & nicht möglich         \\
            \cline{2-4}     & Symbol löschen, Liste filtern\tnote{1} \ccgray                                   & nicht möglich \ccgray & nicht möglich \ccgray \\
            \hline \hline
            \trrr{Enter} & \tbbr{Formular senden / ohne Form \\ zum nächsten Feld springen}         & nicht möglich         & nicht möglich         \\
            \cline{2-4}  & \tbbr{Formular senden / ohne Form \\ zum nächsten Feld springen} \ccgray & nicht möglich \ccgray & nicht möglich \ccgray \\
            \hline \hline
            \trrrr{Scroll} & Browser-Default\tnote{2}                                                        & Browser-Default\tnote{2}                      & Browser-Default\tnote{2} \\
            \cline{2-4}    & \tbbr{\emph{Aussen}: Liste bleibt offen \\ \emph{Innen}: Werte bewegen} \ccgray & \tbbr{\emph{Innen}: \\ Werte bewegen} \ccgray & \tbbr{\emph{Innen}: \\ Werte bewegen} \ccgray \\
            \hline
            \trr{Click} & \tbbr{\emph{in Feld}: - \\ \emph{Pfeil}: Liste gefiltert öffnen\tnote{1}}                        & Liste öffnen                                                  & Liste öffnen \\
            \cline{2-4} & \tbbr{\emph{Innen}: Wert wählen, \\ Liste schliessen \\ \emph{Aussen}: Liste schliessen} \ccgray & \tbbr{\emph{Innen}: Wert wählen, \\ Liste schliessen} \ccgray & \tbbr{\emph{Innen}: Wert \\ wählen / abwählen} \ccgray \\
            \hline 
        \end{tabular}
        \begin{tablenotes}
            \scriptsize
            \item[1] Filter: Leerschlag als normales Symbol gezählt; Liste verändern je nach Anzahl passender Werte; nicht scrollbar
            \item[2] Gleiche Aktion wie wenn die Komponente nicht fokussiert ist; Aktion vom Browser bzw. Betriebssystem vorgegeben
        \end{tablenotes}
    \end{threeparttable}
\end{table}
