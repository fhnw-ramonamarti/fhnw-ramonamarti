\renewcommand{\colwidth}{0.2\textwidth} 
\begin{table}[!htb]
    \tablesettings{Vergleich Interaktion Datalist \& Select in Edge (Windows)}
    \label{table:interactionEdge}
    \footnotesize
    \begin{adjustbox}{max width=\textwidth}
        \begin{threeparttable}
            \begin{tabular}{ l || l | l | l }
                \trrr{\bf{Kriterium}} & \bf{Datalist} & \bf{Select}   & \trrr{\bf{Multiselect}} \\
                \cline{2-3}           & geschlossen   & geschlossen   &  \\
                \cline{2-3}           & offen \ccgray & offen \ccgray &  \\
                \hline \hline
                \trr{$\uparrow$ / $\downarrow$} & Liste öffnen             & Selektion ändern         & \trr{Selektion ändern} \\
                \cline{2-3}                     & Highlight ändern \ccgray & Selektion ändern \ccgray &  \\
                \hline
                \trr{$\leftarrow$ / $\rightarrow$} & Cursor\tnote{1} \ bewegen         & Selektion ändern & \trr{-} \\
                \cline{2-3}                        & Cursor\tnote{1} \ bewegen \ccgray & - \ccgray   &  \\
                \hline
                \trrr{Buchstaben} & Schreiben \& Liste öffnen                   & \tbbr{Selektion auf Such-\\ 
                                                                                        ergebnis\tnote{3} \ ändern}         & \trbbr{3}{\colwidth}{Selektion aufheben \& Selektion auf Suchergebnis\tnote{3} \ ändern} \\
                \cline{2-3}       & Schreiben \& Liste filtern\tnote{2} \ccgray & \tbbr{Selektion auf Such-\\ 
                                                                                        ergebnis\tnote{3} \ ändern} \ccgray & \\
                \hline
                \trr{Leerschlag} & Schreiben \& Liste öffnen                     & Liste öffnen & \trr{-} \\
                \cline{2-3}      & Schreiben \& Liste filtern\tnote{2} \ccgray   & - \ccgray    & \\
                \hline
                \trr{Backspace} & \tbbr{Löschen \& Liste öffnen \\ 
                                        (wenn Feld nicht leer)} & -         & \trr{-} \\
                \cline{2-3}     & Löschen \& Liste filtern\tnote{2} \ccgray & - \ccgray & \\
                \hline
                \trr{Esc}   & -                        & -                        & \trr{-} \\
                \cline{2-3} & Liste schliessen \ccgray & Liste schliessen \ccgray & \\
                \hline \hline
                \trrr{Enter} & \tbbr{\emph{in Formular}: senden \\ 
                                     \emph{sonst}: -}                      & Liste öffnen                    & \trr{-} \\
                \cline{2-3}  & \tbbr{\emph{mit Highlight}: ändern \\ 
                                     \emph{ohne}: Form senden / -} \ccgray & \tbbr{Selektion ändern \& \\ 
                                                                                   Liste schliessen} \ccgray &  \\
                \hline
                \trr{Tab}   & Input-Feld verlassen         & Input-Feld verlassen     & \trr{-} \\
                \cline{2-3} & Input-Feld verlassen \ccgray & Liste schliessen \ccgray & \\
                \hline
                \trrr{\tbbr{PageUp /\\ PageDown}} & Fenster scrollen                       & \tbbr{Selektion auf jeden \\ 
                                                                                                   3. Wert ändern}                  & \trbbr{3}{\colwidth}{Selektion auf vorherige/ nächste \emph{size}\tnote{4} \ Stelle ändern} \\
                \cline{2-3}                       & \tbbr{Highlight auf View-Rand \\ 
                                                          dann seitenweise ändern} \ccgray & \tbbr{Selektion auf View-Rand \\ 
                                                                                                   dann seitenweise ändern} \ccgray & \\
                \hline
                \trrr{Home / End} & \tbbr{Highlight auf ersten/ \\ 
                                          letzten Wert ändern}         & \tbbr{Selektion auf ersten/ \\ 
                                                                               letzten Wert ändern}         & \trbbr{3}{\colwidth}{Selektion auf ersten/ letzten Wert ändern} \\
                \cline{2-3}       & \tbbr{Highlight auf ersten/ \\ 
                                          letzten Wert ändern} \ccgray & \tbbr{Selektion auf ersten/ \\ 
                                                                               letzten Wert ändern} \ccgray & \\
                \hline \hline
                \trrr{Scroll} & Fenster scrollen                                   & Fenster scrollen                           & \trbbr{3}{\colwidth}{\emph{Innen}: Liste scrollen} \\
                \cline{2-3}   & \tbbr{\emph{Aussen}: Liste \texttt{fixed} \tnote{5} \ offen \\ 
                                     \emph{Innen}: Liste scrollen} \ccgray         & \tbbr{\emph{Aussen}: Liste schliessen \\ 
                                                                                           \emph{Innen}: Liste scrollen} \ccgray & \\
                \hline
                \trr{Hover} & -                        & UI-Anpassung             & \trr{-} \\
                \cline{2-3} & Highlight ändern \ccgray & Highlight ändern \ccgray & \\
                \hline
                \trrr{Click} & Liste öffnen             & Liste öffnen                    & \trbbr{3}{\colwidth}{Selektion aufheben \& Selektion ändern} \\
                \cline{2-3}  & Selektion ändern \ccgray & \tbbr{Selektion ändern \& \\ 
                                                                Liste schliessen} \ccgray & \\
                \hline
            \end{tabular}
            \begin{tablenotes}
                \scriptsize
                \item[1] Blinkender Strich in einem Eingabefeld
                \item[2] Filter: Input-Begriffe \emph{and}-verknüpft enthalten; 
                                Liste verändern je nach Anzahl passender Werte
                \item[3] Suche: Erster zu der Eingabe passender Wert aus der Liste, wenn Eingabe nicht passend $\Rightarrow$ letzter noch übereinstimmender Wert; 
                                Liste unverändert; nach Pause / Debounce-Ablauf $\Rightarrow$ neue Suche
                \item[4] Wert von dem size-Attribut
                \item[5] Verhalten der CSS-Position \texttt{fixed}
            \end{tablenotes}
        \end{threeparttable}
    \end{adjustbox}
\end{table}
