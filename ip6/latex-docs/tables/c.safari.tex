\begin{table}[ht!]
    \label{table:interactionSafari}
    \tablesettings{Vergleich Interaktion Datalist \& Select in Safari (Mac)}
    \footnotesize
    \begin{threeparttable}
        \begin{tabular}{ l || l | l | l }
            \trrr{\bf{Kriterium}} & \bf{Datalist} & \bf{Select}   & \trrr{\bf{Multiselect}} \\
            \cline{2-3}           & geschlossen   & geschlossen   &  \\
            \cline{2-3}           & offen \ccgray & offen \ccgray &  \\
            \hline \hline
            \trr{$\uparrow$ / $\downarrow$} & -                        & Liste öffnen             & \trr{Wert ändern} \\
            \cline{2-3}                     & Highlight ändern \ccgray & Highlight ändern \ccgray &  \\
            \hline
            \trr{$\leftarrow$ / $\rightarrow$} & Cursor bewegen\tnote{1}         & -         & \trr{-} \\
            \cline{2-3}                        & Cursor bewegen\tnote{1} \ccgray & - \ccgray &  \\
            \hline 
            \trrr{Buchstaben} & Schreiben, Liste gefiltert öffnen\tnote{2} & \tbbr{Wert ändern auf \\ Suchergebnis\tnote{3}}              & \trrr{\tbbr{Auswahl aufheben, \\ Wert ändern auf \\ Suchergebnis\tnote{3}}} \\
            \cline{2-3}       & Schreiben, Liste filtern\tnote{2} \ccgray  & \tbbr{Highlight ändern auf \\ Suchergebnis\tnote{3}} \ccgray & \\
            \hline
            \trr{Leerschlag} & Schreiben, Liste gefiltert öffnen\tnote{2} & Liste öffnen        & \trr{-} \\
            \cline{2-3}      & Schreiben, Liste filtern\tnote{2} \ccgray  & Wert wählen \ccgray & \\
            \hline
            \trr{Backspace} & \tbbr{Symbol löschen, Liste \\ gefiltert öffnen\tnote{2}} & -                                  & \trr{-} \\
            \cline{2-3}     & Symbol löschen, Liste filtern\tnote{2} \ccgray            & Highlight zu oberstem Wert \ccgray & \\
            \hline
            \trr{Esc}   & -         & -                        & \trr{-} \\
            \cline{2-3} & - \ccgray & Liste schliessen \ccgray & \\
            \hline \hline
            \trrr{Enter} & \tbbr{Formular senden / \\ ohne Form nichts}          & -                                               & \trr{-} \\
            \cline{2-3}  & \tbbr{Highlight wählen, \\ Liste schliessen } \ccgray & \tbbr{Wert wählen, \\ Liste schliessen} \ccgray &  \\
            \hline
            \trr{Tab}   & Input-Feld wechseln         & Input-Feld wechseln & \trr{-} \\
            \cline{2-3} & Input-Feld wechseln \ccgray & - \ccgray           & \\
            \hline
            \trrr{\tbbr{PageUp / \\ PageDown \\ (fn \& $\uparrow$ / $\downarrow$)}} & Fenster scrollen         & Fenster scrollen                               & \trrr{\tbbr{Wert an nächster \\ $size - 1$ wählen \\ \scriptsize{(size = Attribut)}}} \\
            \cline{2-3}                                                             & Fenster scrollen \ccgray & \tbbr{Erster / letzter \\ Wert wählen} \ccgray & \\
            \hline
            \trr{\tbbr{Home / End \\ (fn \& $\leftarrow$ / $\rightarrow$)}} & Fenster scrollen         & Fenster scrollen                     & \trr{\tbbr{Erster / letzter \\ Wert wählen}} \\
            \cline{2-3}                                                     & Fenster scrollen \ccgray & Erster / letzter Wert wählen \ccgray & \\
            \hline \hline
            \trrrr{Scroll} & Browser-Default\tnote{4}                                                                                      & Browser-Default\tnote{4}                                                                & \trrrr{\tbbr{\textit{Innen}: Werte \\ bewegen ohne \\ Highlight / Wert \\ zu ändern}} \\
            \cline{2-3}    & \tbbr{\textit{Aussen}: Liste schliessen \\ \textit{Innen}: Werte bewegen \\ ohne Highlight zu ändern} \ccgray & \tbbr{\textit{Aussen}: - \\ \textit{Innen}: Werte bewegen, \\ Highlight ändern} \ccgray & \\
            \hline
            \trr{Hover} & -         & -                        & \trr{-} \\
            \cline{2-3} & - \ccgray & Highlight ändern \ccgray & \\
            \hline
            \trr{Click} & Liste öffnen        & Liste öffnen                                    & \trr{\tbbr{Auswahl aufheben, \\ Wert wählen}} \\
            \cline{2-3} & Wert wählen \ccgray & \tbbr{Wert wählen, \\ Liste schliessen} \ccgray & \\
            \hline \hline
            \trr{\tbbr{Cmd \& Z\\ (Undo)}} & Browser-Default\tnote{4}         & Browser-Default\tnote{4}         & \trr{Browser-Default\tnote{4}} \\
            \cline{2-3}                    & Browser-Default\tnote{4} \ccgray & Browser-Default\tnote{4} \ccgray & \\
            \hline
            \trr{\tbbr{Cmd \& Shift \\ \& Z (Redo)}} & Browser-Default\tnote{4}         & Browser-Default\tnote{4}         & \trr{Browser-Default\tnote{4}} \\
            \cline{2-3}                              & Browser-Default\tnote{4} \ccgray & Browser-Default\tnote{4} \ccgray & \\
            \hline 
        \end{tabular}
        \begin{tablenotes}
            \scriptsize
            \item[1] Hier: blinkender Strich in einem Eingabefeld
            \item[2] Filter: Leerschlag als normales Symbol gezählt; Liste verändern je nach Anzahl passender Werte
            \item[3] Suche: Erster zu der Eingabe passender Wert aus der Liste, wenn Eingabe nicht passend $\Rightarrow$ sortiert nachfolgender Wert; 
                            Liste unverändert; nach Pause / Debounce-Ablauf $\Rightarrow$ neue Suche
            \item[4] Gleiche Aktion wie wenn die Komponente nicht fokussiert ist; Aktion vom Browser bzw. Betriebssystem vorgegeben
        \end{tablenotes}
    \end{threeparttable}
\end{table}
