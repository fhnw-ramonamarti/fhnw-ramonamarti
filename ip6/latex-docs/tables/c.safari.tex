\begin{table}[ht!]
    \tablesettings{Vergleich Interaktion Datalist \& Select in Safari (Mac)}
    \footnotesize
    \begin{threeparttable}
        \begin{tabular}{ l || l | l | l }
            \trrr{\bf{Kriterium}} & \bf{Datalist} & \bf{Select}   & \trrr{\bf{Multiselect}} \\
            \cline{2-3}           & geschlossen   & geschlossen   &  \\
            \cline{2-3}           & offen \ccgray & offen \ccgray &  \\
            \hline \hline
            \trr{$\uparrow$ / $\downarrow$} & -                           & Liste öffnen                & \trr{Wert wechseln} \\
            \cline{2-3}                     & Fokus-Wert wechseln \ccgray & Fokus-Wert wechseln \ccgray &  \\
            \hline
            \trr{$\leftarrow$ / $\rightarrow$} & Cursor bewegen         & -         & \trr{-} \\
            \cline{2-3}                        & Cursor bewegen \ccgray & - \ccgray &  \\
            \hline
            \trrr{Buchstaben} & Schreiben, Liste gefiltert öffnen\tnote{1} & \tbbr{Wert ändern auf \\ Suchergebnis\tnote{2}}               & \trrr{\tbbr{Auswahl aufheben \\ \& Wert ändern \\ auf Suchergebnis\tnote{2}}} \\
            \cline{2-3}       & Schreiben, Liste filtern\tnote{1} \ccgray  & \tbbr{Fokus-Wert ändern auf \\ Suchergenbis\tnote{2}} \ccgray & \\
            \hline
            \trr{Leerschlag} & Schreiben, Liste gefiltert öffnen\tnote{1} & Liste öffnen        & \trr{-} \\
            \cline{2-3}      & Schreiben, Liste filtern\tnote{1} \ccgray  & Wert wählen \ccgray & \\
            \hline
            \trr{Backspace} & \tbbr{Symbol löschen \& Liste \\ gefiltert öffnen\tnote{1}} & -                             & \trr{-} \\
            \cline{2-3}     & Symbol löschen \& Liste filtern\tnote{1} \ccgray            & Fokus zu obestem Wert \ccgray & \\
            \hline
            \trr{Esc}   & -         & -                        & \trr{-} \\
            \cline{2-3} & - \ccgray & Liste schliessen \ccgray & \\
            \hline \hline
            \trrr{Enter} & \tbbr{Formular senden / \\ ohne Form nichts}                    & -                                                 & \trr{-} \\
            \cline{2-3}  & \tbbr{Fokussierter Wert wählen \& \\ Liste schliessen } \ccgray & \tbbr{Wert wählen \& \\ Liste schliessen} \ccgray &  \\
            \hline
            \trr{Tab}   & Inputfeld wechseln         & Inputfeld wechseln & \trr{-} \\
            \cline{2-3} & Inputfeld wechseln \ccgray & - \ccgray          & \\
            \hline
            \trrr{\tbbr{PageUp / \\ PageDown \\ (fn \& $\uparrow$ / $\downarrow$)}} & Fenster scrollen         & Fenster scrollen                               & \trrr{\tbbr{Wert an nächster \\ $size - 1$ wählen \\ \scriptsize{(size = Attribut)}}} \\
            \cline{2-3}                                                             & Fenster scrollen \ccgray & \tbbr{Erster / letzter \\ Wert wählen} \ccgray & \\
            \hline
            \trr{\tbbr{Home / End \\ (fn \& $\leftarrow$ / $\rightarrow$)}} & Fenster scrollen         & Fenster scrollen                     & \trr{\tbbr{Erster / letzter \\ Wert wählen}} \\
            \cline{2-3}                                                     & Fenster scrollen \ccgray & Erster / letzter Wert wählen \ccgray & \\
            \hline \hline
            \trrrr{Scroll} & Browser-Default                                                                                          & Browser-Default                                                                            & \trrrr{\tbbr{\textit{Innen}: Werte \\ bewegen ohne \\ Fokus / Wert \\ zu ändern}} \\
            \cline{2-3}    & \tbbr{\textit{Aussen}: Liste schliesst \\ \textit{Innen}: Werte bewegen \\ ohne Fokus zu ändern} \ccgray & \tbbr{\textit{Aussen}: nichts \\ \textit{Innen}: Werte bewegen \& \\ Fokus ändern} \ccgray & \\
            \hline
            \trr{Hover} & -         & -                      & \trr{-} \\
            \cline{2-3} & - \ccgray & Fokus wechseln \ccgray & \\
            \hline
            \trr{Click} & Liste öffnen        & Liste öffnen                                      & \trr{\tbbr{Auswahl aufheben \\ \& Wert wählen}} \\
            \cline{2-3} & Wert wählen \ccgray & \tbbr{Wert wählen \& \\ Liste schliessen} \ccgray & \\
            \hline \hline
            \trr{\tbbr{Cmd \& Z\\ (Undo)}} & Browser-Default         & Browser-Default         & \trr{Browser-Default} \\
            \cline{2-3}                    & Browser-Default \ccgray & Browser-Default \ccgray & \\
            \hline
            \trr{\tbbr{Cmd \& Shift \\ \& Z (Redo)}} & Browser-Default         & Browser-Default         & \trr{Browser-Default} \\
            \cline{2-3}                              & Browser-Default \ccgray & Browser-Default \ccgray & \\
            \hline
        \end{tabular}
        \begin{tablenotes}
            \scriptsize
            \item[1] Filter: Leerschlag als normales Symbol gezählt; 
                            Liste verändern je nach Anzahl passender Werte
            \item[2] Suche: Erster zu der Eingabe passender Wert aus der Liste, wenn Eingabe nicht passend $\Rightarrow$ sortiert nachfolgender Wert; 
                            Liste unverändert; nach Pause / Debounce-Ablauf $\Rightarrow$ neue Suche
        \end{tablenotes}
    \end{threeparttable}
\end{table}
